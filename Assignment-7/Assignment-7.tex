\documentclass{assignment}
\ProjectInfos{高等热力学与统计物理}{PHYS2110}{2020-2021学年第二学期}{习题 VII}{截止时间:2021. 4. 20(周二)}{陈稼霖}{45875852}

\begin{document}
\begin{prob}
    根据集团积分的定义和图形展开的规则验证 Mayer 第一定理至 $y^3$ 项.
\end{prob}
\begin{pf}
    Mayer 第一定理:
    \begin{align}
        \mathcal{Q}=\prod_le^{Vb_ly^l},
    \end{align}
    其中
    \begin{align}
        b_l(V,T)=&\frac{1}{Vl!}\int\prod_{i=1}^l\mathrm{d}^3\vec{r}_i\sum_{\text{all $l$-cluster}}\left(\prod f_{ij}\right),\\
        b_1(V,T)=&\frac{1}{V}\int\mathrm{d}^3\vec{r}=1,\\
        b_2(V,T)=&\frac{1}{2!V}\int\mathrm{d}^3\vec{r}_1\vec{r}_2f_{12},\\
        b_3(V,T)=&\frac{1}{3!V}\int\mathrm{d}^3\vec{r}_1\vec{r}_2\vec{r}_3(f_{12}f_{13}+f_{12}f_{23}+f_{13}f_{23}+f_{12}f_{13}f_{23})=\frac{1}{3!V}\int\mathrm{d}\vec{r}_1\mathrm{d}\vec{r}_2\mathrm{d}\vec{r}_3(3f_{12}f_{13}+f_{12}f_{13}f_{23}),\\
        \cdots&
    \end{align}
    故
    \begin{align}
        \notag\mathcal{Q}=&e^{Vb_1y}\times e^{Vb_2y^2}\times e^{Vb_3y^3}\times\cdots=e^{Vy}\times e^{Vb_2y^2}\times e^{Vb_3y^3}\times\cdots\\
        \notag=&\left[1+Vy+\frac{1}{2}(Vy)^2+\frac{1}{3!}(Vy)^3+\cdots\right]\times\left[1+Vb_2y^2+\cdots\right]\times\left[1+Vb_3y^3+\cdots\right]\times\cdots\\
        =&1+Vy+\left(\frac{1}{2}V^2+Vb_2\right)y^2+\left(\frac{1}{3!}V^3+V^2b_2+Vb_3\right)y^3+O(y^4).
    \end{align}
\end{pf}

\begin{prob}
    考虑在长度为 $L$ 的一维线性匣子内的气体系统. 两原子的相互作用能量是 $u_{ij}$
    \begin{align}
        u_{ij}=\left\{\begin{array}{ll}
            \infty,&\abs{x_{ij}}\leq d\\
            0,&\abs{x_{ij}}>d
        \end{array}\right.
    \end{align}
    计算这系统的前两个 virial 系数,并同准确的状态方程
    \begin{align}
        \frac{P}{kT}=\frac{\rho}{1-\rho d}
    \end{align}
    相比较. 其中线密度 $\rho=N/L$.
\end{prob}
\begin{sol}
    
\end{sol}

\begin{prob}
    证明直径为 $d$ 的硬球的三维经典气体的状态方程是
    \begin{align}
        \frac{P}{kT}=\rho\left[1+\frac{2}{3}\pi\rho d^3+\frac{5}{18}\pi^2(\rho d^3)^2+O(\rho^3d^9)\right]
    \end{align}
    试比较同一系统由 Van der Waals 方程给出的 $(\rho d^3)^2$ 的系数.
\end{prob}
\begin{pf}

\end{pf}
\end{document}