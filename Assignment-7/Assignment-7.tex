\documentclass{assignment}
\ProjectInfos{高等热力学与统计物理}{PHYS2110}{2020-2021学年第二学期}{习题 VII}{截止时间:2021. 4. 20(周二)}{陈稼霖}{45875852}

\begin{document}
\begin{prob}
    根据集团积分的定义和图形展开的规则验证 Mayer 第一定理至 $y^3$ 项.
\end{prob}
\begin{pf}
    
\end{pf}

\begin{prob}
    考虑在长度为 $L$ 的一维线性匣子内的气体系统. 两原子的相互作用能量是 $u_{ij}$
    \begin{align}
        u_{ij}=\left\{\begin{array}{ll}
            \infty,&\abs{x_{ij}}\leq d\\
            0,&\abs{x_{ij}}>d
        \end{array}\right.
    \end{align}
    计算这系统的前两个 virial 系数,并同准确的状态方程
    \begin{align}
        \frac{P}{kT}=\frac{\rho}{1-\rho d}
    \end{align}
    相比较. 其中线密度 $\rho=N/L$.
\end{prob}
\begin{sol}
    
\end{sol}

\begin{prob}
    证明直径为 $d$ 的硬球的三维经典气体的状态方程是
    \begin{align}
        \frac{P}{kT}=\rho\left[1+\frac{2}{3}\pi\rho d^3+\frac{5}{18}\pi^2(\rho d^3)^2+O(\rho^3d^9)\right]
    \end{align}
    试比较同一系统由 Van der Waals 方程给出的 $(\rho d^3)^2$ 的系数.
\end{prob}
\begin{pf}

\end{pf}
\end{document}