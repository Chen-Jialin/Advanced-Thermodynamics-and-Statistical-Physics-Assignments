\documentclass{assignment}
\ProjectInfos{高等热力学与统计物理}{PHYS2110}{2020-2021学年第二学期}{习题 VII}{截止时间:2021. 4. 20(周二)}{陈稼霖}{45875852}

\begin{document}
\begin{prob}
    根据集团积分的定义和图形展开的规则验证 Mayer 第一定理至 $y^3$ 项.
\end{prob}
\begin{pf}
    Mayer 第一定理:
    \begin{align}
        \mathcal{Q}=\prod_le^{Vb_ly^l},
    \end{align}
    其中
    \begin{align}
        b_l(V,T)=&\frac{1}{Vl!}\int\prod_{i=1}^l\mathrm{d}^3\vec{r}_i\sum_{\text{all $l$-cluster}}\left(\prod f_{ij}\right),\\
        b_1(V,T)=&\frac{1}{V}\int\mathrm{d}^3\vec{r}=1,\\
        b_2(V,T)=&\frac{1}{2!V}\int\mathrm{d}^3\vec{r}_1\vec{r}_2f_{12},\\
        b_3(V,T)=&\frac{1}{3!V}\int\mathrm{d}^3\vec{r}_1\vec{r}_2\vec{r}_3(f_{12}f_{13}+f_{12}f_{23}+f_{13}f_{23}+f_{12}f_{13}f_{23})=\frac{1}{3!V}\int\mathrm{d}\vec{r}_1\mathrm{d}\vec{r}_2\mathrm{d}\vec{r}_3(3f_{12}f_{13}+f_{12}f_{13}f_{23}),\\
        \cdots&
    \end{align}
    故
    \begin{align}
        \notag\mathcal{Q}=&e^{Vb_1y}\times e^{Vb_2y^2}\times e^{Vb_3y^3}\times\cdots=e^{Vy}\times e^{Vb_2y^2}\times e^{Vb_3y^3}\times\cdots\\
        \notag=&\left[1+Vy+\frac{1}{2}(Vy)^2+\frac{1}{3!}(Vy)^3+\cdots\right]\times\left[1+Vb_2y^2+\cdots\right]\times\left[1+Vb_3y^3+\cdots\right]\times\cdots\\
        =&1+Vy+\left(\frac{1}{2}V^2+Vb_2\right)y^2+\left(\frac{1}{3!}V^3+V^2b_2+Vb_3\right)y^3+O(y^4).
    \end{align}
\end{pf}

\begin{prob}
    考虑在长度为 $L$ 的一维线性匣子内的气体系统. 两原子的相互作用能量是 $u_{ij}$
    \begin{align}
        u_{ij}=\left\{\begin{array}{ll}
            \infty,&\abs{x_{ij}}\leq d\\
            0,&\abs{x_{ij}}>d
        \end{array}\right.
    \end{align}
    计算这系统的前两个 virial 系数,并同准确的状态方程
    \begin{align}
        \frac{P}{kT}=\frac{\rho}{1-\rho d}
    \end{align}
    相比较. 其中线密度 $\rho=N/L$.
\end{prob}
\begin{sol}
    根据两原子的相互作用能量 $u_{ij}$ 的表达式,有
    \begin{align}
        f_{ij}=e^{-\beta u_{ij}}-1=\left\{\begin{array}{ll}
            -1,&\abs{x_{ij}}\leq d\\
            0,&\abs{x_{ij}}>d
        \end{array}\right.
    \end{align}
    根据 Mayer 第一定理,
    \begin{align}
        \label{2-P/kT}
        \frac{P}{kT}=&\sum_{l=1}^{\infty}b_ly^l=b_1y+b_2y^2+b_3y^3+O(y^4),\\
        \label{2-rho}
        \rho=&y\frac{\partial\left(\frac{P}{kT}\right)}{\partial y}=b_1y+2b_2y^2+3b_3y^3+O(y^4),
    \end{align}
    其中一维情况下
    \begin{align}
        b_l(V,T)=&\frac{1}{l!L}\int\prod_{i=1}^l\mathrm{d}x_i\,\sum_{\text{all $l$-clusters}}\left(\prod f_{ij}\right),\\
        b_1(V,T)=&\frac{1}{L}\int\mathrm{d}x_1=1,\\
        b_2(V,T)=&\frac{1}{2!L}\int\mathrm{d}x_1\mathrm{d}x_2\,f_{12}=\frac{1}{2!L}\int\mathrm{d}x_1\int_{-d}^d\mathrm{d}x_{12}\,f_{12}(x_{12})=-d,\\
        \notag b_3(V,T)=&\frac{1}{3!L}\int\mathrm{d}x_1\mathrm{d}x_2\mathrm{d}x_3\,(3f_{12}f_{13}+f_{21}f_{13}f_{23})\\
        \notag=&\frac{1}{3!L}\int\mathrm{d}x_1\int_{-d}^d\mathrm{d}x_{12}\int_{-d}^d\mathrm{d}x_{13}\,(3f_{12}(x_{12})f_{13}(x_1-x_3)+f_{21}(x_{12})f_{13}(x_{13})f_{23}(\abs{x_{12}-x_{13}}))=\frac{3d^2}{2},
    \end{align}
    % 故
    % \begin{align}
    %     \frac{P}{kT}=&y-dy^2+\frac{3d^2}{2}y^3+O(y^4),\\
    %     \rho=&y-2dy^2+\frac{9d^2}{2}y^3+O(y^4).
    % \end{align}
    由式 \eqref{2-rho} 得
    \begin{align}
        \label{2-y}
        y=\rho-2b_2y^2-3b_3y^3+O(y^4).
    \end{align}
    将上式代入式 \eqref{2-P/kT} 中得
    \begin{align}
        \label{2-P/kT-2}
        \frac{P}{kT}=\rho-b_2y^2-2b_3y^3+O(y^4).
    \end{align}
    将式 \eqref{2-y} 代入式 \eqref{2-y} 并近似到 $2$ 阶得
    \begin{align}
        y=\rho-2b_2\rho^2+O(\rho^3).
    \end{align}
    将上式代入式 \eqref{2-P/kT-2} 中可得
    \begin{align}
        \notag\frac{P}{kT}=&\rho-b_2(\rho-2b_2\rho^2)^2-2b_3(\rho-2b_2\rho^2)^3+O(\rho^4)=\rho-b_2\rho^2+(4b_2^2-2b_3)\rho^3+O(\rho^4)\\
        =&\rho\left[1-\frac{1}{2}\beta_1\rho-\frac{2}{3}\beta_2\rho^2+O(\rho^3)\right],
    \end{align}
    其中
    \begin{align}
        \beta_1=&2b_2=-2d,\\
        \beta_2=&3(b_3-2b_2^2)=-\frac{3d^2}{2}.
    \end{align}
    由准确的状态方程出发,有
    \begin{align}
        \frac{P}{kT}=\frac{\rho}{1-\rho d}=\rho\left[1+\rho d+(\rho d)^2+O(\rho^3)\right]=\rho\left[1-\frac{1}{2}\beta_1\rho-\frac{2}{3}\beta_2\rho^2+O(\rho^3)\right].
    \end{align}
    其中的 virial 系数与前一种方法得到的 virial 系数相同.
\end{sol}

\begin{prob}
    证明直径为 $d$ 的硬球的三维经典气体的状态方程是
    \begin{align}
        \frac{P}{kT}=\rho\left[1+\frac{2}{3}\pi\rho d^3+\frac{5}{18}\pi^2(\rho d^3)^2+O(\rho^3d^9)\right]
    \end{align}
    试比较同一系统由 Van der Waals 方程给出的 $(\rho d^3)^2$ 的系数.
\end{prob}
\begin{pf}
    直径为 $d$ 的硬球的三维经典气体的原子间相互作用能量为
    \begin{align}
        u_{ij}=\left\{\begin{array}{ll}
            \infty,&r\leq d\\
            0,&r>d
        \end{array}\right.,
    \end{align}
    从而
    \begin{align}
        f_{ij}=e^{-\beta u_{ij}}-1=\left\{\begin{array}{ll}
            -1,&r\leq d\\
            0,&r>d
        \end{array}\right.
    \end{align}
    利用 Mayer 第一定理对 $\frac{P}{kT}$ 按 $\rho$ 的幂级数展开,有
    \begin{align}
        \frac{P}{kT}=\rho\left[1-\frac{1}{2}\beta_2\rho-\frac{2}{3}\beta_2\rho^2+O(\rho^3)\right],
    \end{align}
    其中 virial 系数
    \begin{align}
        \beta_1=&2b_2,\\
        \beta_2=&3(b_3-2b_2^2),
    \end{align}
    而
    \begin{align}
        b_2=&\frac{1}{2!V}\int\mathrm{d}^3\vec{r}_2\int\mathrm{d}^3r_1\,f_{12}(\abs{\vec{r}_1-\vec{r}_2})=2\pi\int_0^d\mathrm{d}r_{12}\,r_{12}^2f_{12}(r_{12})=-\frac{2\pi d^3}{3},\\
        \notag b_3=&\frac{1}{3!V}\int\mathrm{d}^3\vec{r}_3\int\mathrm{d}^3\vec{r}_2\int\mathrm{d}^3\vec{r}_1\,[3f_{12}(\vec{r}_1-\vec{r}_2)f_{13}(\abs{\vec{r}_1-\vec{r}_3})-f_{12}(\abs{\vec{r}_1-\vec{r}_2})f_{13}(\abs{\vec{r}_1-\vec{r}_3})f_{23}(\abs{\vec{r}_2-\vec{r}_3})]\\
        \notag=&8\pi^2\int_0^d\mathrm{d}r_{12}\,r_{12}^2f(r_{12})\int_0^d\mathrm{d}r_{13}\,r_{13}^2f(r_{13})-\frac{1}{6}\int\mathrm{d}^3\vec{r}_{12}\int\mathrm{d}^3\vec{r}_{13}\,f_{12}(r_{12})f_{13}(r_{13})f_{23}(\abs{\vec{r}_{12}-\vec{r}_{13}})\\
        =&\frac{8\pi^2d^6}{9}-\frac{1}{6}\int\mathrm{d}^3\vec{r}_{12}\int\mathrm{d}^3\vec{r}_{13}\,f_{12}(r_{12})f_{13}(r_{13})f_{23}(\abs{\vec{r}_{12}-\vec{r}_{13}}).
    \end{align}
    其中 $-\int\mathrm{d}^3\vec{r}_{13}\,f_{12}(r_{12})f_{13}(r_{13})f_{23}(\abs{\vec{r}_{12}-\vec{r}_{13}})$ 等价于两个直径 $d$,球心相距 $r_{12}$ 的球的重叠部分体积,即
    \begin{align}
        \notag-\int\mathrm{d}^3\vec{r}_{13}\,f_{12}(r_{12})f_{13}(r_{13})f_{23}(\abs{\vec{r}_{12}-\vec{r}_{13}})=&2\int_0^{d-\frac{r_{12}}{2}}\pi[d^2-(d-x)^2]\,\mathrm{d}x\\
        =&\frac{\pi}{12}(16d^3-12d^2r_{12}+r_{12}^3).
    \end{align}
    从而
    \begin{gather}
        b_3=\frac{8\pi^2d^6}{9}+\frac{\pi^2}{18}\int_0^d\mathrm{d}r_{12}\,r_{12}^2(16d^3-12d^2r_{12}+r_{12}^3)=\frac{8\pi^2d^6}{9}+\frac{5\pi^2d^6}{36},\\
        \Longrightarrow\frac{P}{kT}=\rho\left[1+\frac{2\pi}{3}\rho d^3+\frac{5}{18}\pi^2(\rho d^3)^2+O(\rho^3d^9)\right].
    \end{gather}

    由于将分子作为相互之间无吸引力的硬球处理,该系统的 Van der Waals 方程为
    \begin{gather}
        P(V-Nb)=NkT,\\
        \Longrightarrow\frac{P}{kT}=\frac{N}{V-Nb}=\frac{\rho}{1-b\rho}=\rho+b\rho^2+b^2\rho^3+O(\rho^4),
    \end{gather}
    两种方式得到的 $(\rho d^3)^2$ 的系数均不依赖于温度.
    比较得
    \begin{align}
        b=\frac{2\pi}{3}d^3.
    \end{align}
\end{pf}
\end{document}