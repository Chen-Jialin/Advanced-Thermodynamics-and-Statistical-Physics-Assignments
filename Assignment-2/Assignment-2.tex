\documentclass{assignment}
\ProjectInfos{高等热力学与统计物理}{PHYS2110}{2020-2021学年第二学期}{习题 II}{截止时间:2021. 3. 16(周二)}{陈稼霖}{45875852}

\begin{document}
\begin{prob}
    假设热容量 $C_V$ 为常数,求体积为 $V$,温度为 $T$ 的理想气体的熵(确定到一个任意常数)\\
    用所得结果和 Joule 自由膨胀实验论证熵增加原理.
\end{prob}
\begin{sol}
    对于一般系统,熵的全微分可表为
    \begin{align}
        \mathrm{d}S=\frac{C_V}{T}\,\mathrm{d}T+T\left(\frac{\partial P}{\partial T}\right)_V\mathrm{d}V.
    \end{align}
    对上式积分得
    \begin{align}
        \label{1-S}
        S=\int\left[\frac{C_V}{T}\,\mathrm{d}T+\left(\frac{\partial p}{\partial T}\right)_V\mathrm{d}V\right]+S_0,
    \end{align}
    其中 $S_0$ 为一常数.
    对于理想气体,有状态方程
    \begin{align}
        pV=nRT,
    \end{align}
    故
    \begin{align}
        \left(\frac{\partial p}{\partial T}\right)_V=\frac{nR}{V}.
    \end{align}
    将上式代入式 \eqref{1-S},并且假设定容热容量 $C_V$ 为常数,积分得到理想气体的熵为
    \begin{align}
        S=C_V\ln T+nR\ln V+S_0.
    \end{align}
\end{sol}

\begin{prob}
    某一物质具有下列性质:
    \begin{itemize}
        \item[(i)] 在恒定温度 $T_0$ 下体积从 $V_0$ 膨胀到 $V$ 所做的功为
        \[
            W=RT_0\ln\frac{V}{V_0}.
        \]
        \item[(ii)] 该物质的熵由下式给出
        \[
            S=R\frac{V}{V_0}\left(\frac{T}{T_0}\right)^a.
        \]
        其中 $T_0$, $V_0$ 和 $a$ 为固定常数.
    \end{itemize}
    \begin{itemize}
        \item[1)] 计算该物质的 Helmholtz 自由能.
        \item[2)] 求该物质的状态方程.
        \item[3)] 求在任意恒定温度 $T$ 下体积从 $V_0$ 膨胀到 $V$ 所做的功.
    \end{itemize}
\end{prob}
\begin{sol}
    \begin{itemize}
        \item[1)] 
        \item[2)] 
        \item[3)] 
    \end{itemize}
\end{sol}

\begin{prob}
    一热机循环如右边的 $T-S$ 图所示. 其中 $A$ 代表灰色区域的面积,$B$ 代表灰色区域一下至坐标轴的面积.
    \begin{itemize}
        \item[1)] 证明此热机循环的效率不可能超过可逆循环的效率.
        \item[2)] 证明可逆热机的效率不可能超过工作于最高和最低温度,$T_{\max}$ 和 $T_{\min}$,之间的 Carnot 热机效率.
    \end{itemize}
\end{prob}
\begin{pf}
    \begin{itemize}
        \item[1)] 
        \item[2)] 
    \end{itemize}
\end{pf}

\begin{prob}
    从最小 Gibbs 势的原理而不用 Helmholtz 自由能推导气-液相变的 Maxwell 法则.
\end{prob}
\begin{pf}
    
\end{pf}
\end{document}