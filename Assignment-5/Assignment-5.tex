\documentclass{assignment}
\ProjectInfos{高等热力学与统计物理}{PHYS2110}{2020-2021学年第二学期}{习题 V}{截止时间:2021. 4. 14(周二)}{陈稼霖}{45875852}

\begin{document}
\begin{prob}
    证明体积为 $V$,温度为 $T$ 的辐射场有以下关系:
    \begin{gather*}
        E=V\frac{\pi^2(kT)^4}{15(\hbar c)^3}\\
        F=-\frac{1}{3}E\\
        S=\frac{4}{3}\frac{E}{T}\\
        P=\frac{1}{3}\frac{E}{V}
    \end{gather*}
\end{prob}
\begin{pf}
    辐射场即光子气体,光子是玻色子,达到平衡后遵从玻色分布. 由于容器壁不断发射和吸收光子,光子数不守恒,在导出其分布时,不存在对总粒子数的约束条件,故仅有一个拉格朗日乘子,自由能 $\mu=0$. 辐射场的巨配分函数为
    \begin{align}
        \mathcal{Q}=\prod_{\alpha}\sum_{n_{\alpha}=1}^{\infty}e^{-\beta n_{\alpha}\varepsilon_{\alpha}}=\prod_{\alpha}\frac{1}{1-e^{-\beta\varepsilon_{\alpha}}}.
    \end{align}
    光子的能量可表为
    \begin{align}
        \varepsilon=cp=c\hbar K.
    \end{align}
    考虑到光子的自旋量子数为 $1$,自旋在动量方向的投影可取 $\pm\hbar$ 两个值,相当于左右圆偏振,对光子所有状态的求和可表为
    \begin{align}
        \notag V\int_0^{\infty}\mathrm{d}\varepsilon\,D(\varepsilon)(\cdots)=&2V\int\frac{\mathrm{d}^3\vec{K}}{(2\pi)^3}(\cdots)=\frac{2V}{(2\pi)^3}\int_0^{2\pi}\mathrm{d}\varphi\int_0^{\pi}\sin\theta\,\mathrm{d}\theta\int_0^{\infty}\mathrm{d}K\,K^2(\cdots)\\
        =&\frac{V}{\pi^2}\int_0^{\infty}\mathrm{d}K\,K^2(\cdots)=\frac{V}{\pi^2\hbar^3c^3}\int_0^{\infty}\mathrm{d}\varepsilon\,\varepsilon^2(\cdots)
    \end{align}
    从而得到态密度
    \begin{align}
        D(\varepsilon)=\frac{\varepsilon^2}{\pi^2\hbar^3c^3}.
    \end{align}
    内能为
    \begin{align}
        \notag E=&-\left(\frac{\partial}{\partial\beta}\ln\mathcal{Q}\right)=\sum_{\alpha}\frac{\varepsilon_{\alpha}}{e^{\beta\varepsilon_{\alpha}}-1}=V\int_0^{\infty}\mathrm{d}\varepsilon\frac{D(\varepsilon)\varepsilon}{e^{\beta\varepsilon}-1}=\frac{V}{\pi^2\hbar^3c^3}\int_0^{\infty}\mathrm{d}\varepsilon\frac{\varepsilon^3}{e^{\beta\varepsilon}-1}\\
        \notag&(\text{令 }x=\beta\varepsilon=\frac{\varepsilon}{kT})\\
        =&\frac{V(kT)^4}{\pi^2\hbar^3c^3}\int_0^{\infty}\mathrm{d}x\frac{x^3}{e^x-1}=\frac{V(kT)^4}{\pi^2\hbar^3c^3}\times\frac{\pi^4}{15}=V\frac{\pi^2(kT)^4}{15(\hbar c)^3}.
    \end{align}
    压强为
    \begin{align}
        \notag P=&\frac{kT\ln Q}{V}=\frac{kT}{V}\sum_{\alpha}\ln\left(\frac{1}{1-e^{-\beta\varepsilon_{\alpha}}}\right)=-kT\int_0^{\infty}\mathrm{d}\varepsilon\,D(\varepsilon)\ln(1-e^{-\beta\varepsilon})=-\frac{kT}{\pi^2\hbar^3c^3}\int_0^{\infty}\mathrm{d}\varepsilon\,\varepsilon^2\ln(1-e^{-\beta\varepsilon})\\
        \notag=&-\frac{kT}{\pi^2\hbar^3c^3}\int_0^{\infty}\ln(1-e^{-\beta\varepsilon})\,\mathrm{d}\left(\frac{\varepsilon^3}{3}\right)=-\frac{kT}{\pi^2\hbar^3c^3}\frac{1}{3}\left\{\left.\left[\varepsilon^3\ln(1-e^{-\beta\varepsilon})\right]\right\rvert_0^{\infty}-\int_0^{\infty}\varepsilon^3\,\mathrm{d}\ln(1-e^{-\beta\varepsilon})\right\}\\
        \notag=&\frac{1}{\pi^2\hbar^3c^3}\frac{1}{3}\int_0^{\infty}\mathrm{d}\varepsilon\frac{\varepsilon^3}{e^{\beta\varepsilon}-1}=\frac{1}{3}\frac{E}{V}.
    \end{align}
    熵为
    \begin{align}
        S=k(\ln\mathcal{Q}+\beta U)=\frac{PV}{T}+\frac{E}{T}=\frac{4}{3}\frac{E}{T}.
    \end{align}
    Helmholtz 自由能为
    \begin{align}
        F=U-TS=-\frac{1}{3}E.
    \end{align}
\end{pf}

\begin{prob}
    考虑两维自旋为零的自由 Boson 系统
    \begin{itemize}
        \item[1)] 推导单位面积的态密度公式;
        \item[2)] 推导粒子数密度(面密度)用温度和易逸度表达的公式;
        \item[3)] 证明此系统无凝聚现象.
    \end{itemize}
\end{prob}
\begin{sol}
    \begin{itemize}
        \item[1)] 两维自旋为零的自由 Boson 系统中的单个粒子的能量可表为
        \begin{align}
            \varepsilon=\frac{\hbar^2K^2}{2m}.
        \end{align}
        对所有状态的求和可表为
        \begin{align}
            A\int\frac{\mathrm{d}^2\vec{K}}{(2\pi)^2}=\frac{A}{(2\pi)^2}\int_0^{2\pi}\mathrm{d}\theta\int_0^{\infty}K\,\mathrm{d}K=\frac{A}{2\pi}\int_0^{\infty}\frac{\sqrt{2m\varepsilon}}{\hbar}\mathrm{d}\left(\frac{\sqrt{2m\varepsilon}}{\hbar}\right)=\frac{Am}{2\pi\hbar^2}\int_0^{\infty}\mathrm{d}\varepsilon=A\int_0^{\infty}D(\varepsilon)\,\mathrm{d}\varepsilon,
        \end{align}
        其中 $A$ 为系统的面积,
        从而得到态密度
        \begin{align}
            D(\varepsilon)=\frac{m}{2\pi\hbar^2}.
        \end{align}
        \item[2)] 粒子数面密度为
        \begin{align}
            \notag\rho=&\int_0^{\infty}\frac{D(\varepsilon)}{e^{\beta(\varepsilon-\mu)}-1}\,\mathrm{d}\varepsilon=\frac{m}{2\pi\hbar^2}\int_0^{\infty}\mathrm{d}\varepsilon\frac{ze^{-\beta\varepsilon}}{1-ze^{-\beta\varepsilon}}\\
            \notag&(\text{令 }x=\beta\varepsilon=\frac{\varepsilon}{kT})\\
            \notag=&\frac{mkT}{2\pi\hbar^2}\int_0^{\infty}\frac{ze^{-x}}{1-ze^{-x}}=\frac{mkT}{2\pi\hbar^2}\left.\ln(1-ze^{-x})\right\rvert_0^{\infty}=\frac{mkT}{2\pi\hbar^2}\ln\frac{1}{1-z}=\frac{1}{\lambda^2}\ln\frac{1}{1-z}.
        \end{align}
        其中 $\lambda=\sqrt{\frac{2\pi}{mkT}}\hbar$.
        \item[3)] 对于玻色子,$\mu\leq 0\Longrightarrow 0\leq z=e^{\beta\mu}\leq 1\Longrightarrow$对任意温度和粒子数密度,均存在
        \begin{align}
            z=1-e^{-\rho z^2},
        \end{align}
        即,以上模型对任意温度和粒子数密度均成立,而不存在温度的下限或粒子数密度的上限,故该系统无凝聚现象.
    \end{itemize}
\end{sol}

\begin{prob}
    证明在高温或低密度区域($\rho\lambda^3\ll 1$),自旋为 $j$ 的非相对论自由量子气体的状态方程和熵由下列两式给出:
    \begin{gather*}
        PV=NkT\left[1\pm\frac{\rho\lambda^3}{2^{5/2}(2j+1)}+\cdots\right]\\
        S=Nk\ln\frac{(2j+1)e^{\frac{5}{2}}}{\rho\lambda^3}\pm Nk\frac{\rho\lambda^3}{2^{\frac{7}{2}}(2j+1)}+\cdots
    \end{gather*}
    其中上边的符号对应于 Fermions,下边的符号对应于 Bosons,$\lambda$ 为热波长,$\cdots$ 代表 $\rho\lambda^3$ 的更高阶项.
\end{prob}
\begin{pf}
    
\end{pf}
\end{document}