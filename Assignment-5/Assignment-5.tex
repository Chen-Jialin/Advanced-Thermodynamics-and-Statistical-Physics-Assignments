\documentclass{assignment}
\ProjectInfos{高等热力学与统计物理}{PHYS2110}{2020-2021学年第二学期}{习题 V}{截止时间:2021. 4. 14(周二)}{陈稼霖}{45875852}

\begin{document}
\begin{prob}
    证明体积为 $V$,温度为 $T$ 的辐射场有以下关系:
    \begin{gather*}
        E=V\frac{\pi^2(kT)^4}{15(\hbar c)^3}\\
        F=-\frac{1}{3}E\\
        S=\frac{4}{3}\frac{E}{T}\\
        P=\frac{1}{3}\frac{E}{V}
    \end{gather*}
\end{prob}
\begin{pf}
    
\end{pf}

\begin{prob}
    考虑两维自旋为零的自由 Boson 系统
    \begin{itemize}
        \item[1)] 推导单位面积的态密度公式;
        \item[2)] 推导粒子数密度(面密度)用温度和易逸度表达的公式;
        \item[3)] 证明此系统无凝聚现象.
    \end{itemize}
\end{prob}
\begin{sol}
    
\end{sol}

\begin{prob}
    证明在高温或低密度区域($\rho\lambda^3\ll 1$),自旋为 $j$ 的非相对论自由量子气体的状态方程和熵由下列两式给出:
    \begin{gather*}
        PV=NkT\left[1\pm\frac{\rho\lambda^3}{2^{5/2}(2j+1)}+\cdots\right]\\
        S=Nk\ln\frac{(2j+1)e^{\frac{5}{2}}}{\rho\lambda^3}\pm Nk\frac{\rho\lambda^3}{2^{\frac{7}{2}}(2j+1)}+\cdots
    \end{gather*}
    其中上边的符号对应于 Fermions,下边的符号对应于 Bosons,$\lambda$ 为热波长,$\cdots$ 代表 $\rho\lambda^3$ 的更高阶项.
\end{prob}
\begin{pf}
    
\end{pf}
\end{document}