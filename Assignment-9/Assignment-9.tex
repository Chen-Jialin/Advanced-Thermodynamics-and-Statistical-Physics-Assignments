\documentclass{assignment}
\ProjectInfos{高等热力学与统计物理}{PHYS2110}{2020-2021学年第二学期}{习题 IX}{截止时间:2021. 5. 4(周二)}{陈稼霖}{45875852}

\begin{document}
\begin{prob}
    一稀薄气体处于外力场内,相应的势能为 $V(\vec{r})$. 假设 $V(\vec{r})$ 在分子相互作用力程范围内的变化很小,求出 Boltzmann 方程的近似解并用平均数密度 $n$,平均动量 $\vec{p}_0$ 和 $\vec{p}_0=0$ 时的平均动能表示所得的解.
\end{prob}
\begin{sol}
    势能 $V(\vec{r})$ 分布下的外力场为
    \begin{align}
        \vec{F}=-\vec{\nabla}V(\vec{r}).
    \end{align}
    故 Boltzmann 方程
    \begin{gather}
        \notag\frac{\partial}{\partial t}f(\vec{r},\vec{p};t)+\left(\frac{\vec{p}}{m}\cdot\vec{\nabla}_{\vec{r}}+\vec{F}\cdot\vec{\nabla}_{\vec{p}}\right)f(\vec{r},\vec{p};t)\\
        =\int\mathrm{d}^3\vec{p}_2\,\mathrm{d}^3\vec{p}_1'\,\mathrm{d}^3\vec{p}_2'\abs{T_{if}}^2\delta^4(P'-P)[f(\vec{r},\vec{p}_1';t)f(\vec{r},\vec{p}_2';t)-f(\vec{r},\vec{p};t)f(\vec{r},\vec{p}_2;t)]
    \end{gather}
    可化为
    \begin{gather}
        \notag\frac{\partial}{\partial t}f(\vec{r},\vec{p};t)+\left(\frac{\vec{p}}{m}\cdot\vec{\nabla}_{\vec{r}}-\vec{\nabla}_{\vec{p}}V(\vec{r})\cdot\vec{\nabla}_{\vec{p}}\right)f(\vec{r},\vec{p};t)\\
        =\int\mathrm{d}^3\vec{p}_2\,\mathrm{d}^3\vec{p}_1'\,\mathrm{d}^3\vec{p}_2'\,\abs{T_{if}}^2\delta^4(P'-P)[f(\vec{r},\vec{p}_1';t)f(\vec{r},\vec{p}_2';t)-f(\vec{r},\vec{p};t)f(\vec{r},\vec{p}_2;t)]
    \end{gather}
    设 Boltzmann 方程的试探解为
    \begin{align}
        f(\vec{r},\vec{p};t)=C_1\rho(\vec{r},t)e^{-A(\vec{p}-\vec{p}_0)^2},
    \end{align}
    其中 $A$ 和 $C$ 为待定常数.
    坐标空间的粒子数密度可表为
    \begin{gather}
        \rho(\vec{r})=C_1\rho(\vec{r},t)\int\mathrm{d}^3\vec{p}\,e^{-A(\vec{p}-\vec{p}_0)^2}=C_1\rho(\vec{r},t)\left(\frac{\pi}{A}\right)^{3/2},\\
        \Longrightarrow C_1=\left(\frac{A}{\pi}\right)^{3/2}.
    \end{gather}
    平均动量即为 $\vec{p}_0$:
    \begin{align}
        \langle\vec{p}\rangle=\frac{\int\mathrm{d}^3\vec{p}\,\vec{p}f(\vec{r},\vec{p};t)}{\int\mathrm{d}^3\vec{p}\,f(\vec{r},\vec{p};t)}=\frac{\int\mathrm{d}^3\vec{p}\,\vec{p}e^{-A(\vec{p}-\vec{p}_0)^2}}{\int\mathrm{d}^3\vec{p}\,e^{-A(\vec{p}-\vec{p}_0)^2}}=\vec{p}_0.
    \end{align}
    $\vec{p}_0=0$ 时的平均动能可表为
    \begin{gather}
        \varepsilon=\left\langle\frac{p^2}{2m}\right\rangle=\frac{\int\mathrm{d}^3\vec{p}\,\frac{p^2}{2m}f(\vec{r},\vec{p};t)}{\int\mathrm{d}^3\vec{p}\,f(\vec{r},\vec{p};t)}=\frac{\int\mathrm{d}^3\vec{p}\,\frac{p^2}{2m}e^{-A\vec{p}^2}}{\int\mathrm{d}^3\vec{p}\,e^{-A\vec{p}^2}}=\frac{3}{4Am},\\
        \Longrightarrow A=\frac{3}{4m\varepsilon}.
    \end{gather}
    在这一试探解下,
    \begin{align}
        f(\vec{r},\vec{p}_1';t)f(\vec{r},\vec{p}_2';t)=f(\vec{r},\vec{p};t)f(\vec{r},\vec{p}_2;t),
    \end{align}
    故 Boltzmann 方程右侧等于零,从而
    \begin{gather}
        \frac{\partial}{\partial t}f(\vec{r},\vec{p};t)+\frac{\vec{p}}{m}\cdot\vec{\nabla}_{\vec{r}}f(\vec{r},\vec{p};t)-\vec{\nabla}_{\vec{r}}V(\vec{r})\cdot\vec{\nabla}_{\vec{p}}f(\vec{r},\vec{p};t)=0,\\
        \label{1}\Longrightarrow\frac{\partial}{\partial t}\rho(\vec{r},t)+\frac{\vec{p}}{m}\cdot\vec{\nabla}_{\vec{r}}\rho(\vec{r},t)-\vec{\nabla}_{\vec{r}}V(\vec{r})\cdot\left[-2A(\vec{p}-\vec{p}_0)\right]\rho(\vec{r},t)=0.
    \end{gather}
    当 $\vec{p}_0=0$ 时,可取
    \begin{align}
        \rho(\vec{r})=C_2e^{-2mAV(\vec{r})},
    \end{align}
    从而
    \begin{align}
        f(\vec{r},\vec{p};t)=\frac{3C_2}{4\pi m\varepsilon}\exp\left\{-\frac{3}{2\varepsilon}\left[\frac{(\vec{p}-\vec{p}_0)^2}{2m}+V(\vec{r})\right]\right\}.
    \end{align}
    其中 $C$ 为待定系数.
    对于一般情况,$\vec{p}_0\neq 0$,需作变换
    \begin{align}
        \vec{p}\rightarrow&\vec{p}-\vec{p}_0,\\
        \vec{r}\rightarrow&\vec{r}-\frac{1}{m}\vec{p}_0t
    \end{align}
    以保证式 \eqref{1} 仍然成立.
    此时,
    \begin{align}
        \rho(\vec{r},t)=C_2\exp\left[-2mAV\left(\vec{r}-\frac{1}{m}\vec{p}_0t\right)\right],
    \end{align}
    从而
    \begin{align}
        f(\vec{r},\vec{p};t)=\frac{3C_2}{4\pi m\varepsilon}\exp\left\{-\frac{3}{2\varepsilon}\left[\frac{(\vec{p}-\vec{p}_0)^2}{2m}+V\left(\vec{r}-\frac{1}{m}\vec{p}_0t\right)\right]\right\}.
    \end{align}
    其中 $\varepsilon$ 为 $\vec{p}_0=0$ 时的平均动能,待定系数 $C_2$ 满足归一化条件
    \begin{align}
        \int\mathrm{d}^3\vec{r}\,\mathrm{d}\vec{p}\,f(\vec{r},\vec{p};t)=N.
    \end{align}
\end{sol}

\begin{prob}
    写下一个均匀且无外力作用的气体的 Boltzmann 方程并证明下列 Boltzmann H - 定理:
    \[
        \frac{\mathrm{d}H}{\mathrm{d}t}\leq 0
    \]
    其中
    \[
        H\equiv\int\mathrm{d}^3\vec{p}\,f(\vec{p},t)\ln f(\vec{p},t).
    \]
\end{prob}
\begin{pf}
    对于一个均匀且无外力作用的气体,
    \begin{gather}
        \vec{\nabla}_{\vec{r}}f=0,\\
        \vec{F}=0.
    \end{gather}
    故 Boltzmann 方程
    \begin{gather}
        \notag\frac{\partial}{\partial t}f(\vec{r},\vec{p};t)+\left(\frac{\vec{p}}{m}\cdot\vec{\nabla}_{\vec{r}}+\vec{F}\cdot\vec{\nabla}_{\vec{p}}\right)f(\vec{r},\vec{p};t)\\
        =\int\mathrm{d}^3\vec{p}_2\,\mathrm{d}^3\vec{p}_1'\,\mathrm{d}^3\vec{p}_2'\,\abs{T_{if}}^2\delta^4(P'-P)[f(\vec{r},\vec{p}_1';t)f(\vec{r},\vec{p}_2';t)-f(\vec{r},\vec{p};t)f(\vec{r},\vec{p}_2;t)]
    \end{gather}
    可化为
    \begin{align}
        \label{2-Boltzmann}
        \frac{\partial}{\partial t}f(\vec{p},t)=\int\mathrm{d}^3\vec{p}_2\,\mathrm{d}^3\vec{p}_1'\,\mathrm{d}^3\vec{p}_2'\,\abs{T_{if}}^2\delta^4(P'-P)[f(\vec{p}_1',t)f(\vec{p}_2',t)-f(\vec{p},t)f(\vec{p}_2,t)].
    \end{align}
    $\frac{\mathrm{d}H}{\mathrm{d}t}$ 可表为
    \begin{align}
        \frac{\mathrm{d}H}{\mathrm{d}t}=&\int\mathrm{d}^3\vec{p}\,\frac{\mathrm{d}f(\vec{p},t)}{\mathrm{d}t}[\ln f(\vec{p},t)+1],
    \end{align}
    其中
    \begin{align}
        \int\mathrm{d}^3\vec{p}\frac{\mathrm{d}f(\vec{p},t)}{\mathrm{d}t}=\frac{\mathrm{d}}{\mathrm{d}t}\int\mathrm{d}^3\vec{p}\,f(\vec{p})=\frac{\partial N}{\partial t}=0,
    \end{align}
    故
    \begin{align}
        \frac{\mathrm{d}H}{\mathrm{d}t}=&\int\mathrm{d}^3\vec{p}\,\frac{\mathrm{d}f(\vec{p},t)}{\mathrm{d}t}\ln f(\vec{p},t).
    \end{align}
    将 Boltzmann 方程 \eqref{2-Boltzmann} 代入上式得
    \begin{align}
        \label{2-dH/dt-1}
        \frac{\mathrm{d}H}{\mathrm{d}t}=\int\mathrm{d}^3\vec{p}\,\mathrm{d}^3\vec{p}_2\,\mathrm{d}^3\vec{p}_1'\,\mathrm{d}^3\vec{p}_2'\,\abs{T_{if}}^2\delta(P'-P)[f(\vec{p}_1',t)f(\vec{p}_2',t)-f(\vec{p},t)f(\vec{p}_2,t)]\ln f(\vec{p},t)
    \end{align}
    根据 $H$ 的定义,$f(\vec{p})$ 相对于 $f(\vec{p}_2)$ 并没有特殊性,由于上式中同时对 $\mathrm{d}^3\vec{p}$ 和 $\mathrm{d}^3\vec{p}_2$ 进行积分,根据对称性,交换碰撞的两个粒子的状态,即交换 $\vec{p}$ 和 $\vec{p}_2$,$\vec{p}_1$ 和 $\vec{p}_2$,上面的等式仍成立,
    \begin{align}
        \label{2-dH/dt-2}
        \frac{\mathrm{d}H}{\mathrm{d}t}=\int\mathrm{d}^3\vec{p}_2\,\mathrm{d}^3\vec{p}\,\mathrm{d}^3\vec{p}_2'\,\mathrm{d}^3\vec{p}_1'\,\abs{T_{ij}}^2\delta(P'-P)[f(\vec{p}_2',t)f(\vec{p}_1',t)-f(\vec{p}_2,t)f(\vec{p},t)]\ln f(\vec{p}_2,t)
    \end{align}
    $\frac{\mathrm{d}H}{\mathrm{d}t}$ 又可表为
    \begin{align}
        \label{2-dH/dt-3}
        \notag\frac{\mathrm{d}H}{\mathrm{d}t}=&\frac{1}{2}[\text{式 \eqref{2-dH/dt-1}}+\text{式 \eqref{2-dH/dt-2}}]\\
        =&\frac{1}{2}\int\mathrm{d}^3\vec{p}\,\mathrm{d}^3\vec{p}_2\,\mathrm{d}^3\vec{p}_1'\,\mathrm{d}^3\vec{p}_2'\,\abs{T_{ij}}^2\delta(P'-P)[f(\vec{p}_1',t)f(\vec{p}_2',t)-f(\vec{p},t)f(\vec{p}_2',t)]\ln[f(\vec{p},t)f(\vec{p}_2,t)].
    \end{align}
    根据 $H$ 的定义,$(\vec{p},\vec{p}_2)$ 相对于 $(\vec{p}_1',\vec{p}_2')$ 并没有特殊性,由于上式中同时对 $\mathrm{d}^3\vec{p}$,$\mathrm{d}^3\vec{p}_2$,$\mathrm{d}^3\vec{p}_1'$ 和 $\mathrm{d}^3\vec{p}_2$ 进行积分,根据对称性,交换 $(\vec{p},\vec{p}_2)$ 和 $(\vec{p}_1,\vec{p}_2)$,上面的等式仍然成立,
    \begin{align}
        \label{2-dH/dt-4}
        \notag\frac{\mathrm{d}H}{\mathrm{d}t}=&\frac{1}{2}\int\mathrm{d}^3\vec{p}_1'\,\mathrm{d}^3\vec{p}_2'\,\mathrm{d}^3\vec{p}\,\mathrm{d}^3\vec{p}_2\,\abs{T_{ji}}^2\delta(P-P')[f(\vec{p},t)f(\vec{p}_2,t)-f(\vec{p}_1',t)f(\vec{p}_2',t)]\ln[f(\vec{p}_1',t)f(\vec{p}_2',t)]\\
        =&\frac{1}{2}\int\mathrm{d}^3\vec{p}_1'\,\mathrm{d}^3\vec{p}_2'\,\mathrm{d}^3\vec{p}\,\mathrm{d}^3\vec{p}_2\,\abs{T_{ij}}^2\delta(P'-P)[f(\vec{p},t)f(\vec{p}_2,t)-f(\vec{p}_1',t)f(\vec{p}_2',t)]\ln[f(\vec{p}_1',t)f(\vec{p}_2',t)]
    \end{align}
    $\frac{\mathrm{d}H}{\mathrm{d}t}$ 又可表为
    \begin{align}
        \notag\frac{\mathrm{d}H}{\mathrm{d}t}=&\frac{1}{2}[\text{式 \eqref{2-dH/dt-3}}+\text{式 \eqref{2-dH/dt-4}}]\\
        =&\frac{1}{4}\int\mathrm{d}^3\vec{p}\,\mathrm{d}^3\vec{p}_2\,\mathrm{d}^3\vec{p}_1'\,\mathrm{d}^3\vec{p}_2'\,\abs{T_{ij}}\delta(P'-P)[f(\vec{p}_1',t)f(\vec{p}_2',t)-f(\vec{p},t)f(\vec{p}_2,t)]\ln\frac{f(\vec{p},t)f(\vec{p}_2,t)}{f(\vec{p}_1',t)f(\vec{p}_2',t)}.
    \end{align}
    当 $f(\vec{p}_1',t)f(\vec{p}_2',t)\geq f(\vec{p},t)f(\vec{p}_2,t)$,$[f(\vec{p}_1',t)f(\vec{p}_2',t)-f(\vec{p},t)f(\vec{p}_2,t)]\geq 0$,$\ln\frac{f(\vec{p},t)f(\vec{p}_2,t)}{f(\vec{p}_1',t)f(\vec{p}_2',t)}\leq 0$;当 $f(\vec{p}_1',t)f(\vec{p}_2',t)<f(\vec{p},t)f(\vec{p}_2,t)$,$[f(\vec{p}_1',t)f(\vec{p}_2',t)-f(\vec{p},t)f(\vec{p}_2,t)]<0$,$\ln\frac{f(\vec{p},t)f(\vec{p}_2,t)}{f(\vec{p}_1',t)f(\vec{p}_2',t)}>0$,故
    \begin{align}
        \frac{\mathrm{d}H}{\mathrm{d}t}=\frac{1}{4}\int\mathrm{d}^3\vec{p}\,\mathrm{d}^3\vec{p}_2\,\mathrm{d}^3\vec{p}_1'\,\mathrm{d}^3\vec{p}_2'\,\abs{T_{ij}}\delta(P'-P)[f(\vec{p}_1',t)f(\vec{p}_2',t)-f(\vec{p},t)f(\vec{p}_2,t)]\ln\frac{f(\vec{p},t)f(\vec{p}_2,t)}{f(\vec{p}_1',t)f(\vec{p}_2',t)}\leq 0.
    \end{align}
\end{pf}
\end{document}