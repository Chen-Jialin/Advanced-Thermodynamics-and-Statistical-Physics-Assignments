\documentclass{assignment}
\ProjectInfos{高等热力学与统计物理}{PHYS2110}{2020-2021学年第二学期}{习题 XIV}{截止时间:2021. 6. 8(周二)}{陈稼霖}{45875852}

\begin{document}
\begin{prob}
    设 $\lvert z\rangle=c_0e^{za^{\dagger}}\lvert 0\rangle$ 为 Boson 的相干态,证明
    \begin{itemize}
        \item[i)] 归一化系数 $\abs{c_0}=e^{-\frac{1}{2}\abs{z}^2}$.
        \item[ii)] 粒子数平均值为 $\langle z\rvert a^{\dagger}a\lvert z\rangle=\abs{z}^2$.
        \item[iii)] 求粒子数在 $\lvert z\rangle$ 中分布的涨落.
    \end{itemize}
\end{prob}
\begin{pf}
    \begin{itemize}
        \item[i)] Boson 的相干态可写为
        \begin{align}
            \lvert z\rangle=e^{za^{\dagger}}\lvert 0\rangle=c_0\sum_{n=0}^{\infty}\frac{z^n}{\sqrt{n!}}\lvert n\rangle.
        \end{align}
        由归一化条件
        \begin{align}
            \langle z\vert z\rangle=\abs{c_0}^2\sum_{n=0}^{\infty}\frac{z^{*n}}{\sqrt{n!}}\sum_{m=0}^{\infty}\frac{z^m}{\sqrt{m!}}\langle n\vert m\rangle=\abs{c_0}^2\sum_{n=0}^{\infty}\frac{z^{*n}}{\sqrt{n!}}\sum_{m=0}^{\infty}\frac{z^m}{\sqrt{m!}}\delta_{nm}=\abs{c_0}^2\sum_{n=0}^{\infty}\frac{\abs{z}^{2n}}{n!}=\abs{c_0}^2e^{\abs{z}^2}=1,
        \end{align}
        得归一化系数
        \begin{align}
            \abs{c_0}=e^{-\frac{1}{2}\abs{z}^2}.
        \end{align}
        \item[ii)] 粒子数平均值为
        \begin{align}
            \notag\langle n\rangle=&\langle z\rvert a^{\dagger}a\lvert z\rangle=\abs{c_0}^2\sum_{n=0}^{\infty}\frac{z^{*n}}{\sqrt{n!}}\sum_{m=0}^{\infty}\frac{z^m}{\sqrt{m!}}\langle n\rvert a^{\dagger}a\lvert m\rangle=\abs{c_0}^2\sum_{n=0}^{\infty}\frac{z^{*n}}{\sqrt{n!}}\sum_{m=0}^{\infty}\frac{z^m}{\sqrt{m!}}m\delta_{nm}=\abs{c_0}^2\sum_{n=1}^{\infty}\frac{\abs{z}^{2n}}{(n-1)!}\\
            =&\abs{z}^2\abs{c_0}^2\sum_{n=1}^{\infty}\frac{\abs{z}^{2(n-1)}}{(n-1)!}=\abs{z}^2\abs{c_0}^2\sum_{n=0}^{\infty}\frac{\abs{z}^{2n}}{n!}=\abs{z}^2\abs{c_0}^2e^{\abs{z}^2}=\abs{z}^2.
        \end{align}
        \item[iii)] 粒子数平方的平均值为
        \begin{align}
            \langle n^2\rangle=&\langle z\rvert a^{\dagger}aa^{\dagger}a\lvert z\rangle=\abs{c_0}^2\sum_{n=0}^{\infty}\frac{z^{*n}}{\sqrt{n!}}\sum_{m=0}^{\infty}\frac{z^m}{\sqrt{m!}}\langle n\rvert a^{\dagger}aa^{\dagger}a\lvert m\rangle=\abs{c_0}^2\sum_{n=0}^{\infty}\frac{z^{*n}}{\sqrt{n!}}\sum_{m=0}^{\infty}\frac{z^m}{\sqrt{m!}}m^2\delta_{nm}\\
            =&\abs{c_0}^2\sum_{n=0}^{\infty}\frac{n^2\abs{z}^{2n}}{n!}=\abs{c_0}^2e^{\abs{z}^2}\abs{z}^2(\abs{z}^2+1)=\abs{z}^2(\abs{z}^2+1).
        \end{align}
        粒子数的涨落为
        \begin{align}
            \frac{\Delta n}{\langle n\rangle}=\frac{\sqrt{\langle n^2\rangle-\langle n\rangle^2}}{\langle n\rangle}=\frac{1}{\abs{z}}.
        \end{align}
    \end{itemize}
\end{pf}

\begin{prob}
    一个电子系统由下列 Hamiltonian 描述
    \begin{align}
        \label{2-Kinetic-Energy}
        \mathcal{H}=-\frac{\hbar^2}{2m}\sum_s\int\mathrm{d}^3\,\vec{r}\,\psi_s^{\dagger}(\vec{r})\nabla^2\psi_s(\vec{r})+\frac{1}{2}\sum_{s_1,s_2}\int\mathrm{d}^3\vec{r}_1\,\mathrm{d}^3\vec{r}_2\,\psi_{s_1}^{\dagger}(\vec{r}_1)\psi_{s_2}^{\dagger}(\vec{r}_2)u(\abs{\vec{r}_1-\vec{r}_2})\psi_{s_2}(\vec{r}_2)\psi_{s_1}(\vec{r}_1)
    \end{align}
    其中 $s$ 代表自旋的两个分量. 试写出此 Hamiltonian 用动量-自旋态的湮灭产生算符 $a_{\vec{p},s}$, $a_{\vec{p},s}^{\dagger}$ 和 $u(\abs{\vec{r}_1-\vec{r}_2})$ 的 Fourier 变换 $u_{\vec{q}}=\int\mathrm{d}^3\vec{r}\,e^{-\frac{i}{\hbar}\vec{q}\cdot\vec{r}}u(r)$ 表示的形式.
\end{prob}
\begin{pf}
    动能
    \begin{align}
        K\equiv-\frac{\hbar^2}{2m}\sum_s\int\mathrm{d}^3\vec{r}\,\psi_s^{\dagger}(\vec{r})\nabla^2\psi_s(\vec{r}),
    \end{align}
    其中场算符
    \begin{align}
        \label{2-Field-Op}
        \psi_s(\vec{r})=&\frac{1}{\sqrt{V}}\sum_{\vec{p}}a_{\vec{p},s}e^{\frac{i}{\hbar}\vec{p}\cdot\vec{r}},\\
        \label{2-Field-Op-dagger}
        \psi_s^{\dagger}(\vec{r})=&\frac{1}{\sqrt{V}}\sum_{\vec{p}}a_{\vec{p},s}^{\dagger}e^{-\frac{i}{\hbar}\vec{p}\cdot\vec{r}},
    \end{align}
    而动量算符
    \begin{align}
        \nabla=\frac{i}{\hbar}\hat{\vec{p}}.
    \end{align}
    代入动能(式 \eqref{2-Kinetic-Energy})中可得
    \begin{align}
        K=\sum_s\sum_{\vec{p},\vec{p}'}\frac{p^2}{2m}a_{\vec{p}',s}^{\dagger}a_{\vec{p},s}\frac{1}{V}\int\mathrm{d}\vec{r}\,e^{\frac{i}{\hbar}(\vec{p}-\vec{p}')\cdot\vec{r}}=\sum_s\sum_{\vec{p},\vec{p}'}\frac{p^2}{2m}a_{\vec{p}',s}^{\dagger}a_{\vec{p},s}\delta_{\vec{p},\vec{p}'}=\sum_s\sum_{\vec{p}}\frac{p^2}{2m}a_{\vec{p},s}^{\dagger}a_{\vec{p},s}.
    \end{align}
    相互作用能
    \begin{align}
        \label{2-Interaction-Energy}
        \Omega\equiv\frac{1}{2}\sum_{s_1,s_2}\int\mathrm{d}^3\vec{r}_1\,\mathrm{d}^3\vec{r}_2\,\psi_{s_1}^{\dagger}(\vec{r}_1)\psi_{s_2}(\vec{r}_2)u(\abs{\vec{r}_1-\vec{r}_2})\psi_{s_2}(\vec{r}_2)\psi_{s_1}(\vec{r}_1),
    \end{align}
    其中相互作用势的 Fourier 变换
    \begin{align}
        u(\abs{\vec{r}_1-\vec{r}_2})=\frac{1}{V}\int\mathrm{d}^3\,\vec{q}e^{\frac{i}{\hbar}\vec{q}\cdot\vec{r}}u_{\vec{q}}.
    \end{align}
    将上式和场算符(式 \eqref{2-Field-Op}, \eqref{2-Field-Op-dagger})代入相互作用能(式 \eqref{2-Interaction-Energy})中可得
    \begin{align}
        \notag K=&\frac{1}{2}\sum_{s_1,s_2}\sum_{\vec{p}_1,\vec{p}_1',\vec{p}_2,\vec{p}_2'}a_{\vec{p}_1',s_1}^{\dagger}a_{\vec{p}_2',s_2}^{\dagger}a_{\vec{p}_2,s_2}a_{\vec{p}_1,s_1}\frac{1}{V^2}\int\mathrm{d}^3\vec{r}_1\,e^{\frac{i}{\hbar}(\vec{p}_1-\vec{p}_1')\cdot\vec{r}_1}\int\mathrm{d}^3\vec{r}_2\,e^{\frac{i}{\hbar}(\vec{p}_2-\vec{p}_2')\cdot\vec{r}_2}\int\mathrm{d}^3\vec{q}\,e^{-\frac{i}{\hbar}\vec{q}\cdot\vec{r}}u_{\vec{q}}\\
        \notag=&\frac{1}{2}\sum_{s_1,s_2}\sum_{\vec{p}_1,\vec{p}_1',\vec{p}_2,\vec{p}_2'}a_{\vec{p}_1',s_1}^{\dagger}a_{\vec{p}_2',s_2}^{\dagger}a_{\vec{p}_2,s_2}a_{\vec{p}_1,s_1}\delta_{\vec{p}_1,\vec{p}_1'}\delta_{\vec{p}_2,\vec{p}_2'}\frac{1}{V}\int\mathrm{d}^3\vec{q}\,e^{-\frac{i}{\hbar}\vec{q}\cdot\vec{r}}u_{\vec{q}}\\
        =&\frac{1}{2}\sum_{s_1,s_2}\sum_{\vec{p}_1,\vec{p}_2}a_{\vec{p}_1,s_1}^{\dagger}a_{\vec{p}_2}^{\dagger}a_{\vec{p}_2,s_2}a_{\vec{p}_1,s_1}\frac{1}{V}\int\mathrm{d}^3\vec{q}\,e^{-\frac{i}{\hbar}\vec{q}\cdot\vec{r}}u_{\vec{q}}.
    \end{align}
    因此,该电子系统哈密顿量可写为
    \begin{align}
        \mathcal{H}=\sum_s\sum_{\vec{p}}\frac{p^2}{2m}a_{\vec{p}}^{\dagger}a_{\vec{p}}+\frac{1}{2}\sum_{s_1,s_2}\sum_{\vec{p}_1,\vec{p}_2}a_{\vec{p}_1,s_1}^{\dagger}a_{\vec{p}_2}^{\dagger}a_{\vec{p}_2,s_2}a_{\vec{p}_1,s_1}\frac{1}{V}\int\mathrm{d}^3\vec{q}\,e^{-\frac{i}{\hbar}\vec{q}\cdot\vec{r}}u_{\vec{q}}.
    \end{align}
\end{pf}
\end{document}