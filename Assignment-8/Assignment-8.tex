\documentclass{assignment}
\ProjectInfos{高等热力学与统计物理}{PHYS2110}{2020-2021学年第二学期}{习题 VIII}{截止时间:2021. 4. 27(周二)}{陈稼霖}{45875852}

\begin{document}
\begin{prob}
    Liouville 定理的另一种证法:令 $\omega_t$ 为某一组系统在时间 $t$ 占据的相空间体积. 用正则运动方程证明:
    \[
        \frac{\mathrm{d}\omega_t}{\mathrm{d}t}=0
    \]
    提示:考虑 $\omega_{t+\mathrm{d}t}-\omega_t$.
\end{prob}
\begin{pf}
    设系统的广义坐标为 $\{q_a\}$,广义动量为 $\{p_a\}$. 某组系统在时间 $t$ 占据的的相空间体积为
    \begin{align}
        \omega_t=\int\prod_a\mathrm{d}q_a\mathrm{d}p_a.
    \end{align}
    从时间 $t$ 到时间 $t+\mathrm{d}t$,发生如下演化:
    \begin{align}
        q_a'=&q_a+\dot{q}_a\mathrm{d}t\\
        p_a'=&p_a+\dot{p}_a\mathrm{d}t.
    \end{align}
    该组系统在时间 $t+\mathrm{d}t$ 占据的相空间的体积为
    \begin{align}
        \omega_{t+\mathrm{d}t}=\int\prod_a\mathrm{d}q_a'\mathrm{d}p_a'=\int\prod_aJ\,\mathrm{d}q_a\mathrm{d}p_a,
    \end{align}
    其中雅可比行列式
    \begin{align}
        J=\abs{\frac{\partial(q_1',\cdots;p_1',\cdots)}{\partial(q_1,\cdots;p_1,\cdots)}}=\prod_a\left(1+\frac{\partial\dot{q}_a}{\partial q_a}\right)\left(1+\frac{\partial\dot{p}_a}{\partial p_a}\right)+O(t^2)=1+\sum_a\left(\frac{\partial\dot{q}_a}{\partial q_a}+\frac{\partial\dot{p}_a}{\partial p_a}\right)t+O(t^2).
    \end{align}
    将哈密顿方程代入上式得
    \begin{align}
        J=1+O(t^2).
    \end{align}
    从而
    \begin{gather}
        \omega_{t+\mathrm{d}t}=[1+O(t^2)]\omega_t,\\
        \Longrightarrow\frac{\mathrm{d}\omega}{\mathrm{d}t}=\lim_{\mathrm{d}t\rightarrow 0}\frac{\omega_{t+\mathrm{d}t}-\omega_t}{\mathrm{d}t}=0.
    \end{gather}
\end{pf}

\begin{prob}
    体积为 $V$ 的容器盛有分子数密度为 $\rho$ 的气体. 假设各分子独立地随机运动.
    \begin{itemize}
        \item[(1)] 证明在容器内一小体积 $v\ll V$ 内出现 $n$ 个分子的几率 $P_n$ 近似地由 Poisson 分布给出,即
        \[
            P_n=\frac{1}{n!}(\rho v)^ne^{-\rho v}.
        \]
        \item[(2)] 如果 $\rho=2.5\times 10^{19}\text{cm}^{-3}$,试估计当 $v=1\text{nm}^3$ 和 $v=1\text{cm}^3$ 时 $v$ 内无分子的几率.
    \end{itemize}
\end{prob}
\begin{sol}
    
\end{sol}
\end{document}