% !TEX program = pdflatex -> bibtex -> pdflatex*2
\documentclass{assignment}
\ProjectInfos{高等热力学与统计物理}{PHYS2110}{2020-2021学年第二学期}{习题 I}{截止时间:2021. . (周二)}{陈稼霖}{45875852}

\begin{document}
\begin{prob}
    令变量 $x$, $y$, $z$ 满足方程 $f(x,y,z)=0$,$w$ 为 $x$, $y$, $z$ 中任意两个变量的函数. 证明:
    \begin{itemize}
        \item[1)] $\left(\frac{\partial x}{\partial y}\right)_w\left(\frac{\partial y}{\partial z}\right)_w=\left(\frac{\partial x}{\partial z}\right)_w$
        \item[2)] $\left(\frac{\partial x}{\partial y}\right)_z\left(\frac{\partial y}{\partial x}\right)_z=1$
        \item[3)] $\left(\frac{\partial x}{\partial y}\right)_z\left(\frac{\partial y}{\partial z}\right)_x\left(\frac{\partial z}{\partial x}\right)_y=-1$
    \end{itemize}
    并用理想气体的状态方程验证 2) 和 3),其中 $x=P$, $y=T$, $z=V$.
\end{prob}
\begin{pf}
    \begin{itemize}
        \item[1)] 
        \item[2)] 
        \item[3)] 
    \end{itemize}
\end{pf}

\begin{prob}
    假设热容量 $C_V$ 为常数,证明理想气体绝热过程中的下列关系:
    \begin{itemize}
        \item[1)] $PV^{\gamma}=$常数
        \item[2)] $TV^{\gamma-1}=$常数
        \item[3)] $PT^{\frac{\gamma}{1-\gamma}}=$常数
    \end{itemize}
    其中 $\gamma\equiv C_P/C_V$,并计算理想气体从初态 $(P_1,V_1)$ 到末态 $(P_2,V_2)$ 所做的功.
\end{prob}
\begin{sol}
    \begin{itemize}
        \item[1)] 
        \item[2)] 
        \item[3)] 
    \end{itemize}
\end{sol}

\begin{prob}
    理想气体的 Carnot 循环:$A\rightarrow B\rightarrow C\rightarrow D\rightarrow A$\\
    --- 等温过程:$A\rightarrow B$(与高温热源接触)$C\rightarrow D$(与低温热源接触)\\
    --- 绝热过程:$B\rightarrow C$, $D\rightarrow A$.\\
    假设热容量为常数,证明循环效率为
    \[
        \eta=1-\frac{T_1}{T_2}.
    \]
\end{prob}
\begin{pf}
    
\end{pf}
\end{document}