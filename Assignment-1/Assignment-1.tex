% !TEX program = pdflatex -> bibtex -> pdflatex*2
\documentclass{assignment}
\ProjectInfos{高等热力学与统计物理}{PHYS2110}{2020-2021学年第二学期}{习题 I}{截止时间:2021. . (周二)}{陈稼霖}{45875852}

\begin{document}
\begin{prob}
    令变量 $x$, $y$, $z$ 满足方程 $f(x,y,z)=0$,$w$ 为 $x$, $y$, $z$ 中任意两个变量的函数. 证明:
    \begin{itemize}
        \item[1)] $\left(\frac{\partial x}{\partial y}\right)_w\left(\frac{\partial y}{\partial z}\right)_w=\left(\frac{\partial x}{\partial z}\right)_w$
        \item[2)] $\left(\frac{\partial x}{\partial y}\right)_z\left(\frac{\partial y}{\partial x}\right)_z=1$
        \item[3)] $\left(\frac{\partial x}{\partial y}\right)_z\left(\frac{\partial y}{\partial z}\right)_x\left(\frac{\partial z}{\partial x}\right)_y=-1$
    \end{itemize}
    并用理想气体的状态方程验证 2) 和 3),其中 $x=P$, $y=T$, $z=V$.
\end{prob}
\begin{pf}
    \begin{itemize}
        \item[1)] 将 $w$ 写作 $x$, $y$ 的函数
        \begin{align}
            w=w(x,y)
        \end{align}
        并取微分,有
        \begin{align}
            \label{1-dw}
            \mathrm{d}w=\left(\frac{\partial w}{\partial x}\right)_y\mathrm{d}x+\left(\frac{\partial w}{\partial y}\right)_x\mathrm{d}y.
        \end{align}
        令上式中 $\mathrm{d}w=0$,得
        \begin{align}
            \label{1-dx/dy}
            \left(\frac{\partial x}{\partial y}\right)_w=-\frac{\left(\frac{\partial w}{\partial y}\right)_x}{\left(\frac{\partial w}{\partial x}\right)_y}.
        \end{align}
        对式 $f(x,y,z)=0$ 取微分,有
        \begin{align}
            \label{1-df}
            \mathrm{d}f=\left(\frac{\partial f}{\partial x}\right)_{y,z}\mathrm{d}x+\left(\frac{\partial f}{\partial y}\right)_{x,z}\mathrm{d}y+\left(\frac{\partial f}{\partial z}\right)_{x,y}\mathrm{d}z=0,
        \end{align}
        从而可将 $\mathrm{d}x$ 表示为
        \begin{align}
            \mathrm{d}x=-\frac{\left(\frac{\partial f}{\partial y}\right)_{x,z}\mathrm{d}y+\left(\frac{\partial f}{\partial z}\right)_{x,y}\mathrm{d}z}{\left(\frac{\partial f}{\partial x}\right)_{y,z}}.
        \end{align}
        将上式代入式 \eqref{1-dw},得
        \begin{align}
            \mathrm{d}w=-\left(\frac{\partial w}{\partial x}\right)_y\frac{\left(\frac{\partial f}{\partial y}\right)_{x,z}\mathrm{d}y+\left(\frac{\partial f}{\partial z}\right)_{x,y}\mathrm{d}z}{\left(\frac{\partial f}{\partial x}\right)_{y,z}}+\left(\frac{\partial w}{\partial y}\right)_x\mathrm{d}y.
        \end{align}
        令上式中 $\mathrm{d}w=0$,得
        \begin{align}
            \label{1-dy/dz}
            \left(\frac{\partial y}{\partial z}\right)_w=\frac{-\left(\frac{\partial w}{\partial x}\right)_y\left(\frac{\partial f}{\partial z}\right)_{x,y}}{\left(\frac{\partial w}{\partial x}\right)_y\left(\frac{\partial f}{\partial y}\right)_{x,z}-\left(\frac{\partial w}{\partial y}\right)_x\left(\frac{\partial f}{\partial x}\right)_{y,z}}.
        \end{align}
        由式 \eqref{1-df},也可将 $\mathrm{d}y$ 表为
        \begin{align}
            \mathrm{d}y=-\frac{\left(\frac{\partial f}{\partial x}\right)_{y,z}\mathrm{d}x+\left(\frac{\partial f}{\partial z}\right)_{x,y}\mathrm{d}z}{\left(\frac{\partial f}{\partial y}\right)_{x,z}}.
        \end{align}
        将上式代入式 \eqref{1-dw},得
        \begin{align}
            \mathrm{d}w=\left(\frac{\partial w}{\partial x}\right)_y\mathrm{d}x-\left(\frac{\partial w}{\partial y}\right)_x\frac{\left(\frac{\partial f}{\partial x}\right)_{y,z}\mathrm{d}x+\left(\frac{\partial f}{\partial z}\right)_{x,y}\mathrm{d}z}{\left(\frac{\partial f}{\partial y}\right)_{x,z}}.
        \end{align}
        令上式中 $\mathrm{d}w=0$ 得
        \begin{align}
            \label{1-dx/dz}
            \left(\frac{\partial x}{\partial z}\right)_w=\frac{\left(\frac{\partial w}{\partial y}\right)_x\left(\frac{\partial f}{\partial z}\right)_{x,y}}{\left(\frac{\partial w}{\partial x}\right)_y\left(\frac{\partial f}{\partial y}\right)_{x,z}-\left(\frac{\partial w}{\partial y}\right)_x\left(\frac{\partial f}{\partial x}\right)_{y,z}}
        \end{align}
        式 \eqref{1-dx/dy} 与 \eqref{1-dy/dz} 相乘恰好等于式 \eqref{1-dx/dz},
        \begin{align}
            \left(\frac{\partial x}{\partial y}\right)_w\left(\frac{\partial y}{\partial z}\right)_w=\frac{\left(\frac{\partial w}{\partial y}\right)_x\left(\frac{\partial f}{\partial z}\right)_{x,y}}{\left(\frac{\partial w}{\partial x}\right)_y\left(\frac{\partial f}{\partial y}\right)_{x,z}-\left(\frac{\partial w}{\partial y}\right)_x\left(\frac{\partial f}{\partial x}\right)_{y,z}}=\left(\frac{\partial x}{\partial z}\right)_w,
        \end{align}
        得证.
        \item[2)] 令 $z$ 保持不变,即在式 \eqref{1-df} 中令 $\mathrm{d}z=0$,得
        \begin{align}
            \mathrm{d}f=\left(\frac{\partial f}{\partial x}\right)_{y,z}\mathrm{d}x+\left(\frac{\partial f}{\partial y}\right)_{x,z}\mathrm{d}y=0,
        \end{align}
        从而
        \begin{align}
            \label{1-dx/dy}
            \left(\frac{\partial x}{\partial y}\right)_z=&-\frac{\left(\frac{\partial f}{\partial y}\right)_{x,z}}{\left(\frac{\partial f}{\partial x}\right)_{y,z}},\\
            \left(\frac{\partial y}{\partial x}\right)_z=&-\frac{\left(\frac{\partial f}{\partial x}\right)_{y,z}}{\left(\frac{\partial f}{\partial y}\right)_{x,z}}.
        \end{align}
        上面两式相乘,得
        \begin{align}
            \left(\frac{\partial x}{\partial y}\right)_z\left(\frac{\partial y}{\partial x}\right)_z=1.
        \end{align}
        \item[3)] 令式 \eqref{1-df} 中的 $\mathrm{d}x=0$ ,得
        \begin{align}
            \label{1-dy/dz}
            \left(\frac{\partial y}{\partial z}\right)_x=-\frac{\left(\frac{\partial f}{\partial z}\right)_{x,y}}{\left(\frac{\partial f}{\partial y}\right)_{x,z}}.
        \end{align}
        令式 \eqref{1-df} 中的 $\mathrm{d}y=0$,得
        \begin{align}
            \label{1-dz/dx}
            \left(\frac{\partial z}{\partial x}\right)_y=-\frac{\left(\frac{\partial f}{\partial x}\right)_{y,z}}{\left(\frac{\partial f}{\partial z}\right)_{x,y}}.
        \end{align}
        式 \eqref{1-dx/dy}, \eqref{1-dy/dz} 和 \eqref{1-dz/dx} 相乘,得
        \begin{align}
            \left(\frac{\partial x}{\partial y}\right)_z\left(\frac{\partial y}{\partial z}\right)_x\left(\frac{\partial z}{\partial x}\right)_y=-1.
        \end{align}
    \end{itemize}

    理想气体的状态方程为
    \begin{align}
        PV=nRT.
    \end{align}
    \emph{验证 2)}:
    \begin{align}
        &\left\{\begin{array}{l}
            \left(\frac{\partial P}{\partial T}\right)_V=\frac{nR}{V},\\
            \left(\frac{\partial T}{\partial P}\right)_V=\frac{V}{nR},
        \end{array}\right.\\
        \Longrightarrow&\left(\frac{\partial P}{\partial T}\right)_V\left(\frac{\partial T}{\partial P}\right)_V=1.
    \end{align}
    \emph{验证 3)}:
    \begin{align}
        &\left\{\begin{array}{l}
            \left(\frac{\partial P}{\partial T}\right)_V=\frac{nR}{V},\\
            \left(\frac{\partial T}{\partial V}\right)_P=\frac{P}{nR},\\
            \left(\frac{\partial V}{\partial P}\right)_T=-\frac{nRT}{P^2},
        \end{array}\right.\\
        \Longrightarrow&\left(\frac{\partial P}{\partial T}\right)_V\left(\frac{\partial T}{\partial V}\right)_P\left(\frac{\partial V}{\partial P}\right)_T=-\frac{nRT}{VP}=-1.
    \end{align}
\end{pf}

\begin{prob}
    假设热容量 $C_V$ 为常数,证明理想气体绝热过程中的下列关系:
    \begin{itemize}
        \item[1)] $PV^{\gamma}=$常数
        \item[2)] $TV^{\gamma-1}=$常数
        \item[3)] $PT^{\frac{\gamma}{1-\gamma}}=$常数
    \end{itemize}
    其中 $\gamma\equiv C_P/C_V$,并计算理想气体从初态 $(P_1,V_1)$ 到末态 $(P_2,V_2)$ 所做的功.
\end{prob}
\begin{sol}
    \begin{itemize}
        \item[1)] 
        \item[2)] 
        \item[3)] 
    \end{itemize}
\end{sol}

\begin{prob}
    理想气体的 Carnot 循环:$A\rightarrow B\rightarrow C\rightarrow D\rightarrow A$\\
    --- 等温过程:$A\rightarrow B$(与高温热源接触)$C\rightarrow D$(与低温热源接触)\\
    --- 绝热过程:$B\rightarrow C$, $D\rightarrow A$.\\
    假设热容量为常数,证明循环效率为
    \[
        \eta=1-\frac{T_1}{T_2}.
    \]
\end{prob}
\begin{pf}
    
\end{pf}
\end{document}