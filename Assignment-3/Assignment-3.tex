\documentclass{assignment}
\ProjectInfos{高等热力学与统计物理}{PHYS2110}{2020-2021学年第二学期}{习题 III}{截止时间:2021. 3. 23(周二)}{陈稼霖}{45875852}

\begin{document}
\begin{prob}
    证明大整数 $N$ 阶乘的 Sterling 公式
    \[
        \ln N!=N(\ln N-1)+O(\ln N)
    \]
    其中 $O(\ln N)$ 表示误差与 $\ln N$ 同阶. (一个简单的证明方法是考虑用矩形法近似积分 $\int_1^N\mathrm{d}x\,\ln x$.)
\end{prob}
\begin{pf}
    对于大整数 $N$,
    \begin{align}
        \ln N!=\sum_{n=1}^N\ln n\approx\int_1^N\mathrm{d}x\,\ln x=(x\ln x)_1^{N}-\int_1^Nx\,\mathrm{d}(\ln x)=N\ln N-\int_1^N\mathrm{d}x=N(\ln N-1)+1\approx N(\ln N-1).
    \end{align}
    以上证明了 $\ln N!$ 的近似值,下面来讨论 $\ln N!$ 与其近似值之间的误差 $\ln N!-N(\ln N-1)$. 由于对 $\forall N\geq 4$,
    \begin{gather}
        \left(\frac{N+1}{N}\right)^{N+1}=\left(1+\frac{1}{N}\right)^{N+1}=\left(1+\frac{1}{N}\right)\left(1+\frac{1}{N}\right)^N\leq\left(1+\frac{1}{N}\right)e\leq N\\
        \Longleftrightarrow(N+1)^{N+1}\leq N^{N+2}\\
        \Longleftrightarrow(N+1)\ln(N+1)\leq(N+2)\ln N\\
        \Longleftrightarrow(N+1)\ln(N+1)-N\leq(N+2)\ln N-N\\
        \left[\text{利用}\ln N!=\sum_{n=1}^N\ln N\leq\int_1^{N+1}\mathrm{d}x\,\ln x=(N+1)\ln(N+1)-N\right]\\
        \Longleftrightarrow\ln N!\leq(N+2)\ln N-N\\
        \Longleftrightarrow\ln N!-N(\ln N-1)\leq 2\ln N,
    \end{gather}
    即对 $\forall N\geq 4$,存在 $M=2$,使得 $\ln N!-N(\ln N-1)\leq M\ln N$,即误差与 $\ln N$ 同阶.

    综上,
    \begin{align}
        \ln N!=N(\ln N-1)+O(\ln N).
    \end{align}
\end{pf}

\begin{prob}
    考虑由 $M$ 个相同的 $S$ 系统,$M'$ 个相同的 $S'$ 系统等等组成的一个正则系综. 系综中的系统处在不同的位置但相互热接触. 令系统 $S,S',\cdots$ 的 Hamiltonian 为 $H,H',\cdots$,其本征态和本征能量由下列方程给出
    \begin{gather*}
        H\psi_i=E_j\psi_j\\
        H'\psi_j'=E_j'\psi_j'\\
        \cdots
    \end{gather*}
    证明:找到某一特定系统 $S$ 处于 $\psi_j$ 状态的几率是
    \[
        P_j=\frac{1}{Q}e^{-\beta E_j}.
    \]
    找到某一特定系统 $S'$ 处于 $\psi_j'$ 状态的几率是
    \[
        P_j'=\frac{1}{Q}e^{-\beta E_j'}
    \]
    ……\\
    其中 $Q=\sum_je^{-\beta E_j}$,$Q'=\sum_je^{-\beta E_j'}$,$\cdots$.
\end{prob}
\begin{pf}
    整个系综的 Hamiltonian 为
    \begin{align}
        \mathcal{H}=\sum_{a=1}^MH_a+\sum_{a=1}^{M'}H_a'.
    \end{align}
    整个系综的波函数为
    \begin{align}
        \Psi=\prod_{a=1}^M\psi^{(a)}\prod_{a=1}^{M'}\psi^{'(a)}
    \end{align}
    其中 $\psi^{(j)}$ 为系综中第 $j$ 个 $S$ 系统的本征态,$\psi^{'(j)}$ 为系综中第 $j$ 个 $S'$ 系统的本征态.
    假设系综内系统按能量的分布为 $\{M_j,M_j'\}$,其中 $M_j$ 为处于 $H$ 的本征态 $j$ 的系统数目,$M_j'$ 为处于 $H'$ 的本征态 $j'$ 的系统数目,则系综内 $S$ 系统的总数可表为
    \begin{align}
        \sum_jM_j=&M,\\
        \sum_jM_j'=&M'.
    \end{align}
    系综的总能量可表为
    \begin{align}
        \sum_jM_jE_j+\sum_jM_j'E_j'=\varepsilon.
    \end{align}
    对应于分布 $\{M_j,M_j'\}$ 的系综状态数为
    \begin{align}
        \Omega(\{M_j,M_j'\})=\frac{M!}{\prod_jM_j!}\frac{M'!}{\prod_jM_j'!}.
    \end{align}
    对应于不限定 $S'$ 系统分布,而 $S'$ 系统分布为 $\{M_j\}$ 的系综的状态数为
    \begin{align}
        \Omega(\{M_j\})=\sum_{\{M_j'\}}\Omega(\{M_j,M_j'\}).
    \end{align}
    对应于不限定 $S'$ 系统分布,而 $S'$ 系统分布为 $\{M_j\}$ 的系综的状态数为
    \begin{align}
        \Omega(\{M_j'\})=\sum_{\{M_j\}}\Omega(\{M_j,M_j'\}).
    \end{align}
    给定 $M$, $M'$ 和 $\epsilon$ 下系综状态总数为
    \begin{align}
        \Omega=\sum_{\{M_j,M_j'\}}\Omega(\{M_j,M_j'\}).
    \end{align}
    求 $\Omega(\{M_j,M_j'\})$ 关于 $\{M_j,M_j'\}$ 在固定 $M$, $M'$, $\varepsilon$ 条件下的最大值:
    \begin{align}
        \label{2-parital/partialMj}
        \frac{\partial}{\partial M_j}\left[\ln\Omega(\{M_j,M_j'\})-\alpha_1\sum_jM_j-\alpha_2\sum_jM_j'-\beta\left(\sum_jM_jE_j+\sum_jM_j'E_j'\right)\right]=&0,\\
        \label{2-partial/partialMj'}
        \frac{\partial}{\partial M_j'}\left[\ln\Omega(\{M_j,M_j'\})-\alpha_1\sum_jM_j-\alpha_2\sum_jM_j'-\beta\left(\sum_jM_jE_j+\sum_jM_j'E_j'\right)\right]=&0,
    \end{align}
    其中 $\alpha$ 和 $\beta$ 为 Lagrangian 不定乘子.
    当 $M\rightarrow\infty$ 时,$M_j\rightarrow\infty$,利用 Sterling 公式,有
    \begin{align}
        \ln\Omega(\{M_j\})\approx&M(\ln M-1)-\sum_jM_j(\ln M_j-1)+M'(\ln M'-1)-\sum_jM_j'(\ln M_j'-1).
    \end{align}
    上两式分别代入式 \eqref{2-parital/partialMj} 和 \eqref{2-partial/partialMj'},得
    \begin{align}
        -\ln M_j-\alpha_1-\beta E_j=&0,\\
        -\ln M_j'-\alpha_2-\beta E_j'=&0.
    \end{align}
    故该系综的最可几分布为
    \begin{align}
        M_j=&e^{-\alpha_1-\beta E_j},\\
        M_j'=&e^{-\alpha_2-\beta E_j'}
    \end{align}
    找到某一特定系统 $S$ 处于 $\psi_j$ 状态的几率约等于最可几分布内系统处于本征态 $\psi_j$ 的几率
    \begin{align}
        P_j=\frac{M_j}{M}=\frac{e^{-\alpha_1-\beta E_j}}{\sum_je^{-\alpha_1-\beta E_j}}=\frac{1}{Q}e^{-\beta E_j},
    \end{align}
    其中 $Q=\sum_je^{-\beta E_j}$.
    找到某一特定系统 $S'$ 处于 $\psi_j'$ 状态的几率约等于最可几分布内系统处于本征态 $\psi_j'$ 的几率
    \begin{align}
        P_j'=\frac{M_j'}{M}=\frac{e^{-\alpha_2-\beta E_j}}{\sum_je^{-\alpha_2-\beta E_j'}}=\frac{1}{Q'}e^{-\beta E_j'},
    \end{align}
    其中 $Q'=\sum_je^{-\beta E_j'}$.
\end{pf}

\begin{prob}
    证明巨正则系综的最可几分布内粒子数的涨落为
    \[
        \frac{\Delta N}{\langle N\rangle}=\sqrt{\frac{kT\rho\kappa_T}{\langle N\rangle}}
    \]
    其中 $\rho=\langle N\rangle/V$ 为密度,$\Delta N^2=(N-\langle N\rangle)^2$ 的平均值(均方偏差),$\kappa_T$ 为等温压缩系数. 由此可见 $\kappa_T>0$.
\end{prob}
\begin{pf}
    巨正则系综的最可几分布的平均粒子数为
    \begin{align}
        \langle N\rangle=\sum_N\sum_{j(N)}P_{j(N)}N=\frac{\sum_N\sum_{j(N)}Ne^{-\beta E_{j(N)-\gamma N}}}{\sum_N\sum_{j(N)}e^{-\beta E_{j(N)}-\gamma N}}.
    \end{align}
    注意到
    \begin{align}
        \frac{\partial\langle N\rangle}{\partial\gamma}=&\frac{\partial}{\partial\gamma}\left(\frac{\sum_N\sum_{j(N)}Ne^{-\beta E_{j(N)}-\gamma N}}{\sum_N\sum_{j(N)}e^{-\beta E_{j(N)}-\gamma N}}\right)\\
        =&-\frac{\sum_N\sum_{j(N)}N^2e^{-\beta E_{j(N)}-\gamma N}}{\sum_N\sum_{j(N)}e^{-\beta E_{j(N)}-\gamma N}}+\frac{\left(\sum_N\sum_{j(N)}Ne^{-\beta E_{j(N)}-\gamma N}\right)}{\left(\sum_N\sum_{j(N)}e^{-\beta E_{j(N)}-\gamma N}\right)}\\
        =&-\langle N^2\rangle+\langle N\rangle^2.
    \end{align}
    巨正则系综的最可几分布内粒子数的均方偏差
    \begin{align}
        (\Delta N)^2=\langle(N-\langle N\rangle)^2\rangle=\langle N^2\rangle-\langle N\rangle^2=-\frac{\partial\langle N\rangle}{\partial\gamma}.
    \end{align}
    由于
    \begin{align}
        \gamma=-\frac{\mu}{kT},
    \end{align}
    故有
    \begin{align}
        (\Delta N)^2=-\frac{\partial\langle N\rangle}{\partial\gamma}=kT\left(\frac{\partial\langle N\rangle}{\partial\mu}\right)_{T,V}.
    \end{align}
    巨正则系综的最可几分布内粒子数的涨落为
    \begin{align}
        \label{3-Nfluctuation}
        \frac{\Delta N}{\langle N\rangle}=\sqrt{\frac{kT}{\langle N\rangle^2}\left(\frac{\partial\langle N\rangle}{\partial\mu}\right)_{T,V}}.
    \end{align}
    由于 Gibbs 势的全微分为
    \begin{align}
        \notag\mathrm{d}G=&-S\,\mathrm{d}T+V\,\mathrm{d}P+\mu\,\mathrm{d}N=-(\langle N\rangle s)\,\mathrm{d}T+(\langle N\rangle v)\,\mathrm{d}P+\mu\,\mathrm{d}\langle N\rangle\\
        =&\mathrm{d}(\langle N\rangle\mu)=\langle N\rangle\,\mathrm{d}\mu+\mu\,\mathrm{d}(\langle N\rangle),
    \end{align}
    其中 $s=\frac{S}{\langle N\rangle}$, $v=\frac{V}{\langle N\rangle}$ 分别是单粒子对应的熵和体积,
    化学势的全微分为
    \begin{align}
        \mathrm{d}\mu=v\,\mathrm{d}p-s\,\mathrm{d}T.
    \end{align}
    由上式得,
    \begin{align}
        \left(\frac{\partial\mu}{\partial v}\right)_T=v\left(\frac{\partial p}{\partial v}\right)_T.
    \end{align}
    同时由于 $v=\frac{V}{\langle N\rangle}$,在保持 $V$ 不变而 $\langle N\rangle$ 发生变化的情况下,上式可表为
    \begin{gather}
        \frac{\partial\langle N\rangle}{\partial v}\left(\frac{\partial\mu}{\partial\langle N\rangle}\right)_{T,V}=\left(\frac{\partial v}{\partial\langle N\rangle}\right)^{-1}\left(\frac{\partial\mu}{\partial\langle N\rangle}\right)=-\frac{\langle N\rangle^2}{V}\left(\frac{\partial\mu}{\partial\langle N\rangle}\right)_{T,V}=v\left(\frac{\partial p}{\partial v}\right)_T,\\
        \Longrightarrow\frac{1}{\langle N\rangle^2}\left(\frac{\partial\langle N\rangle}{\partial\mu}\right)_{T,V}=\frac{1}{V}\frac{1}{v}\left(\frac{\partial v}{\partial p}\right)_T.
    \end{gather}
    上式代入式 \eqref{3-Nfluctuation} 得
    \begin{align}
        \frac{\Delta N}{\langle N\rangle}=\sqrt{kT\frac{1}{V}\frac{1}{v}\left(\frac{\partial v}{\partial p}\right)_T}=\sqrt{kT\frac{\rho}{\langle N\rangle}\kappa_T},
    \end{align}
    其中 $\rho=\frac{\langle N\rangle}{V}$ 为粒子数密度,$\kappa_T=\frac{1}{v}\left(\frac{v}{p}\right)_T$ 为等温压缩系数.
\end{pf}
\end{document}