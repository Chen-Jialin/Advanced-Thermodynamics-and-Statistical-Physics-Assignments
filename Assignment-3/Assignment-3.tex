\documentclass{assignment}
\ProjectInfos{高等热力学与统计物理}{PHYS2110}{2020-2021学年第二学期}{习题 III}{截止时间:2021. 3. 23(周二)}{陈稼霖}{45875852}

\begin{document}
\begin{prob}
    证明大整数 $N$ 阶乘的 Sterling 公式
    \[
        \ln N!=N(\ln N-1)+O(\ln N)
    \]
    其中 $O(\ln N)$ 表示误差与 $\ln N$ 同阶. (一个简单的证明方法是考虑用矩形法近似积分 $\int_1^N\mathrm{d}x\,\ln x$.)
\end{prob}
\begin{pf}
    
\end{pf}

\begin{prob}
    考虑由 $M$ 个相同的 $S$ 系统,$M'$ 个相同的 $S'$ 系统等等组成的一个正则系综. 系综中的系统处在不同的位置但相互热接触. 令系统 $S,S',\cdots$ 的 Hamiltonian 为 $H,H',\cdots$,其本征态和本征能量由下列方程给出
    \begin{gather*}
        H\psi_i=E_j\psi_j\\
        H'\psi_j'=E_j'\psi_j'\\
        \cdots
    \end{gather*}
    证明:找到某一特定系统 $S$ 处于 $\psi_j$ 状态的几率是
    \[
        P_j=\frac{1}{Q}e^{-\beta E_j}.
    \]
    找到某一特定系统 $S'$ 处于 $\psi_j'$ 状态的几率是
    \[
        P_j=\frac{1}{Q}e^{-\beta E_j'}
    \]
    ……\\
    其中 $Q=\sum_je^{-\beta E_j}$,$Q'=\sum_je^{-\beta E_j'}$,$\cdots$.
\end{prob}
\begin{pf}

\end{pf}

\begin{prob}
    证明局正则系综的最可几分布内粒子数的涨落为
    \[
        \frac{\Delta N}{\langle N\rangle}=\sqrt{\frac{kT\rho\kappa_T}{\langle N\rangle}}
    \]
    其中 $\rho=\langle N\rangle/V$ 为密度,$\Delta N^2=(N-\langle N\rangle)^2$ 的平均值(均方偏差),$\kappa_T$ 为等温压缩系数. 由此可见 $\kappa_T>0$.
\end{prob}
\begin{pf}
    
\end{pf}
\end{document}