\documentclass{assignment}
\ProjectInfos{高等热力学与统计物理}{PHYS2110}{2020-2021学年第二学期}{习题 X}{截止时间:2021. 5. 9(周日)}{陈稼霖}{45875852}

\begin{document}
\begin{prob}
    令 $\mathcal{Q}_G$ 是具有两体势能为
    \[
        u(r)=\left\{\begin{array}{ll}
            \infty,&\text{如果 }r=0,\\
            u_l,&\text{如果 }r=\text{第 }l\text{ 近邻}
        \end{array}\right.
    \]
    的格气的巨配分函数.\\
    证明:只要令
    \begin{gather*}
        N=N_{\uparrow}\\
        y=\exp\left\{\frac{1}{kT}\left(2\mu H-\sum_ln_l[\varepsilon_l(\uparrow\downarrow)-\varepsilon_l(\uparrow\uparrow)]\right)\right\}\\
        u_l=2[\varepsilon_l(\uparrow\uparrow)-\varepsilon_l(\uparrow\downarrow)]
    \end{gather*}
    则对应于 Ising 模型配分函数 $Q_I$ 格气配分函数可以表示为
    \[
        \mathcal{Q}_{G}=Q_I\exp\left\{\frac{\mathcal{N}}{kT}\left[\mu H+\frac{1}{2}\sum_ln_l\varepsilon_l(\uparrow\uparrow)\right]\right\}
    \]
    其中 $\mathcal{N}$ 是格点总数,$y$ 是格气的易逸度,$\varepsilon_l(\uparrow\uparrow)=\varepsilon_l(\downarrow\downarrow)$ 表示 Ising 模型中第 $l$ 近邻自旋平行的相互作用能,$\varepsilon_l(\uparrow\downarrow)$ 是相应的自旋反平行的作用能,$n_l$ 为每一格点第 $l$ 近邻的数目.
\end{prob}
\begin{pf}
    Ising 模型的能量为
    \begin{align}
        \label{1-UI}
        U_I=\sum_l[N_{\uparrow\uparrow}^l\varepsilon^l(\uparrow\uparrow)+N_{\downarrow\downarrow}\varepsilon^l(\downarrow\downarrow)+N_{\uparrow\downarrow}^l\varepsilon(\uparrow\downarrow)]-\mu H(N_{\uparrow}-N_{\downarrow}),
    \end{align}
    其中 $N_{\uparrow\uparrow}^l$ 为第 $l$ 近邻自旋平行向上的对数,$N_{\downarrow\downarrow}^l$ 为第 $l$ 近邻自旋平行向下的对数,$N_{\uparrow\downarrow}^l$ 为第 $l$ 近邻自旋反平行的对数,$\varepsilon^l(\uparrow\uparrow)=\varepsilon^l(\downarrow\downarrow)$ 为第 $l$ 近邻自旋平行的相互作用能,$\varepsilon^l(\uparrow\downarrow)$ 为第 $l$ 近邻自旋反平行的相互作用能,$\mu$ 为格点的磁矩,$H$ 为沿 $z$ 方向的磁场强度.
    设格点总数为 $\mathcal{N}$,则有
    \begin{align}
        N_{\uparrow}+N_{\downarrow}=\mathcal{N}.
    \end{align}
    设每个格点的第 $l$ 近邻的数目为 $n_l$,令自旋向上的格点向其第 $l$ 近邻连线,总连线数目为 $n_lN_{\uparrow}$,每对第 $l$ 近邻自旋平行向下贡献两条连线,每对第 $l$ 近邻自旋平行向上贡献一条连线,从而有
    \begin{align}
        2N_{\uparrow\uparrow}^l+N_{\uparrow\downarrow}^l=n_lN_{\uparrow},
    \end{align}
    同理有
    \begin{align}
        2N_{\downarrow\downarrow}+N_{\uparrow\downarrow}=n_lN_{\downarrow}.
    \end{align}
    由上面三式,我们可以解得
    \begin{align}
        N_{\downarrow}=&\mathcal{N}-N_{\uparrow},\\
        N_{\uparrow\downarrow}=&n_lN_{\uparrow}-2N_{\uparrow\uparrow}^l,\\
        N_{\downarrow\downarrow}=&\frac{1}{2}n_l\mathcal{N}-n_lN_{\uparrow}+N_{\uparrow\uparrow}^l.
    \end{align}
    将上面三式回代入 Ising 模型能量(式 \eqref{1-UI})得
    \begin{align}
        U_I=2\sum_lN_{\uparrow\uparrow}^l[\varepsilon^l(\uparrow\uparrow)-\varepsilon^l(\uparrow\downarrow)]+N_{\uparrow}^l\left\{\sum_ln_l[\varepsilon_l(\uparrow\downarrow)-\varepsilon_l(\uparrow\uparrow)]-2\mu H\right\}+\mathcal{N}\left[\frac{1}{2}\sum_ln_l\varepsilon_l(\uparrow\uparrow)+\mu H\right]
    \end{align}
    Ising 模型的巨配分函数为
    \begin{align}
        \mathcal{Q}_I=\sum e^{-\frac{U_I}{kT}}.
    \end{align}
    其中求和是对所有自旋分布求和.

    格气的巨配分函数可表为
    \begin{align}
        \mathcal{Q}_G=\sum_Ny^N\frac{1}{N!}\sum_{\text{$N$个可区分粒子的分布}}e^{-\frac{U_g}{kT}}=\sum_Ny^N\sum_{\text{$N$个不可区分粒子的分布}}e^{-\frac{1}{kT}\sum_ln_{pp}^lu_l}.
    \end{align}
    其中 $n_{pp}^l$ 是第 $l$ 近邻的对数.
    当我们令 $N=N_{\uparrow}$,$n_{pp}^l=N_{\uparrow\uparrow}^l$,$y=\exp\left\{\frac{1}{kT}\left(2\mu H-\sum_ln_l[\varepsilon_l(\uparrow\downarrow)-\varepsilon_l(\uparrow\uparrow)]\right)\right\}$,$u_l=2[\varepsilon_l(\uparrow\uparrow)-\varepsilon_l(\uparrow\downarrow)]$ 时,
    \begin{align}
        \notag\mathcal{Q}_G=&\sum_{N_{\uparrow}=1}^{\mathcal{N}}\exp\left\{\frac{N_{\uparrow}}{kT}(2\mu H-\sum_ln_l[\varepsilon_l(\uparrow\downarrow)-\varepsilon_l(\uparrow\uparrow)])\right\}\sum_{\text{$N_{\uparrow}$ 个 $\uparrow$ 的分布}}\exp\left\{-\frac{1}{kT}\sum_lN_{\uparrow\uparrow}^l\cdot 2[\varepsilon_l(\uparrow\uparrow)-\varepsilon_l(\uparrow\downarrow)]\right\}\\
        \notag=&\sum_{\text{$\uparrow$ 的分布}}\exp\left\{-\frac{1}{kT}\left[\sum_lN_{\uparrow\uparrow}^l\cdot 2[\varepsilon_l(\uparrow\uparrow)-\varepsilon_l(\uparrow\downarrow)]-N_{\uparrow}[(\varepsilon_l(\uparrow\uparrow)-\varepsilon_l(\uparrow\downarrow))+2\mu H]\right]\right\}\\
        =&\sum\exp\left\{\frac{\mathcal{N}}{kT}\left[\mu H+\frac{1}{2}\sum_ln_l\varepsilon_l(\uparrow\uparrow)\right]-\frac{1}{kT}U_I\right\}=Q_I\exp\left\{\frac{\mathcal{N}}{kT}\left[\mu H+\frac{1}{2}\sum_ln_l\varepsilon_l(\uparrow\uparrow)\right]\right\}.
    \end{align}
\end{pf}
\end{document}