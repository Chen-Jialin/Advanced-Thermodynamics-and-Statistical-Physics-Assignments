\documentclass{assignment}
\ProjectInfos{高等热力学与统计物理}{PHYS2110}{2020-2021学年第二学期}{习题 X}{截止时间:2021. 5. 9(周二)}{陈稼霖}{45875852}

\begin{document}
\begin{prob}
    令 $\mathcal{Q}_G$ 是具有两体势能为
    \[
        u(r)=\left\{\begin{array}{ll}
            \infty,&\text{如果 }r=0,\\
            u_l,&\text{如果 }r=\text{第 }l\text{ 近邻}
        \end{array}\right.
    \]
    的格气的巨配分函数.\\
    证明:只要令
    \begin{gather*}
        N=N_{\uparrow}\\
        y=\exp\left\{\frac{1}{kT}\left(2\mu H-\sum_ln_l[\varepsilon_l(\uparrow\downarrow)-\varepsilon_l(\uparrow\uparrow)]\right)\right\}\\
        u_l=2[\varepsilon_l(\uparrow\uparrow)-\varepsilon_l(\uparrow\downarrow)]
    \end{gather*}
    则对应于 Ising 模型配分函数 $Q_I$ 格气配分函数可以表示为
    \[
        \mathcal{Q}_{G}=Q_I\exp\left\{\frac{\mathcal{N}}{kT}\left[\mu H+\frac{1}{2}\sum_ln_l\varepsilon_l(\uparrow\uparrow)\right]\right\}
    \]
    其中 $\mathcal{N}$ 是格点总数,$y$ 是格气的易逸度,$\varepsilon_l(\uparrow\uparrow)=\varepsilon_l(\downarrow\downarrow)$ 表示 Ising 模型中第 $l$ 近邻自旋平行的相互作用能,$\varepsilon_l(\uparrow\downarrow)$ 是相应的自旋反平行的作用能,$n_l$ 为每一格点第 $l$ 近邻的数目.
\end{prob}
\begin{sol}
    
\end{sol}
\end{document}