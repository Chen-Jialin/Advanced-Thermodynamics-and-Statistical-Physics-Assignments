\documentclass{assignment}
\ProjectInfos{高等热力学与统计物理}{PHYS2110}{2020-2021学年第二学期}{习题 IV}{截止时间:2021. 3. 30(周二)}{陈稼霖}{45875852}

\begin{document}
\begin{prob}
    计算下列系统内电子气体的 Fermi 能,Fermi 温度和 Fermi 速度.
    \begin{itemize}
        \item[1)] 室温下的金属钠:密度为 $0.97\mathrm{g}/\mathrm{cm}^3$,每个原子贡献一个传导电子,假设它们的能量和动量的关系为 $E=\frac{p^2}{2m}$,(即忽略晶格场对电子运动的影响).
        \item[2)] 天狼星的伴星(白矮星):其质量约为太阳质量的 $0.98$ 倍,半径约为太阳半径的 $0.0084$ 倍. 假设星体全部由氦组成.
    \end{itemize}
\end{prob}
\begin{sol}
    \begin{itemize}
        \item[1)] 电子属于 Fermion,按照自由 Fermi 气体模型,传导电子的数密度可用 Fermi 波矢表示,
        \begin{align}
            \rho=\frac{\omega K_F^3}{6\pi^2},
        \end{align}
        其中金属钠中的传导电子数密度 $\rho=\frac{N}{V}=\frac{N_A}{M_{\text{Na}}/\rho_m}=\frac{6.02\times 10^{23}\text{ mol}^{-1}}{(22.99\text{ g}\cdot\text{mol}^{-1})/(0.97\times 10^6\text{ g}\cdot\text{m}^{-3})}=2.540\times 10^{28}\text{m}^{-3}$,传导电子的各能级的简并度 $\omega=2$,故得 Fermi 波矢为
        \begin{align}
            K_F=9.09\times 10^9\text{ m}^{-1}.
        \end{align}
        Fermi 速度为
        \begin{align}
            v_F=\frac{\hbar K_F}{m}=6.62\times 10^6\text{ m}\cdot\text{s}^{-1},
        \end{align}
        其中电子的质量 $m=9.11\times 10^{-31}$ kg. Fermi 能量为
        \begin{align}
            \varepsilon_F=\frac{\hbar^2K_F^2}{2m}=5.05\times 10^{-19}\text{ J}=3.16\text{ eV}.
        \end{align}
        Fermi 温度为
        \begin{align}
            T_F=\frac{\varepsilon_F}{k}=3.66\times 10^4\text{ K}.
        \end{align}
        \item[2)] 太阳的质量为 $m_{\odot}=1.989\times 10^{30}$ kg,太阳的半径为 $r_{\odot}=6.6934\times 10^8$ m,该白矮星的质量密度为 $\rho_m=\frac{m}{V}=\frac{0.98m_{\odot}}{\frac{4}{3}\pi(0.0084r_{\odot})^3}=2.62\times 10^9\text{ kg}\cdot\text{m}^{-3}$. 假设白矮星中电子全部从原子内挤出,每个氦原子挤出 $\chi=2$ 个电子,白矮星的电子数密度为 $\rho=\frac{\chi N_A}{M_{\text{He}}/\rho_m}=\frac{2\times 6.02\times 10^{23}\text{ mol}^{-1}}{(4.00\times 10^{-3}\text{ kg}\cdot\text{mol}^{-1})/(2.62\times 10^9\text{ kg}\cdot\text{m}^{-3})}=7.89\times 10^{35}\text{ m}^{-3}$.
        对于氦原子Fermi 波矢为
        \begin{align}
            K_F=\left(\frac{6\pi^2\rho}{\omega}\right)^{1/3}=2.86\times 10^{12}\text{ m}^{-1}.
        \end{align}
        Fermi 能量为
        \begin{align}
            \varepsilon_F=c\sqrt{p_F^2+m^2c^2}-mc^2=c\sqrt{\hbar^2k_F^2+m^2c^2}-mc^2=4.01\times 10^{-14}\text{ J}=2.51\times 10^5\text{ eV}.
        \end{align}
        Fermi 速度为
        \begin{align}
            v_F=\frac{p_Fc^2}{\varepsilon_F^2+mc^2}=\frac{c^2\hbar K_F}{\varepsilon_F+mc^2}=2.22\times 10^8\text{ m}\cdot\text{s}^{-1}.
        \end{align}
        Fermi 温度为
        \begin{align}
            T_F=\frac{\varepsilon_F}{k}=2.91\times 10^9\text{ K}.
        \end{align}
    \end{itemize}
\end{sol}

\begin{prob}
    证明非相对论简并电子气体的热力学函数是
    \begin{gather*}
        G=N\mu=N\varepsilon_F\left[1-\frac{1}{12}\pi^2\left(\frac{kT}{\varepsilon_F}\right)^2-\frac{1}{80}\pi^4\left(\frac{kT}{\varepsilon_F}\right)^4+\cdots\right]\\
        E=\frac{3}{5}N\varepsilon_F\left[1+\frac{5}{12}\pi^2\left(\frac{kT}{\varepsilon_F}\right)^2-\frac{1}{16}\pi^4\left(\frac{kT}{\varepsilon_F}\right)+\cdots\right]\\
        C_V=\frac{1}{2}N\pi^2\frac{k^2T}{\varepsilon_F}\left[1-\frac{3}{10}\pi^2\left(\frac{kT}{\varepsilon_F}\right)^2+\cdots\right]\\
        S=\frac{1}{2}N\pi^2\frac{k^2T}{\varepsilon_F}\left[1-\frac{1}{10}\left(\frac{kT}{\varepsilon_F}\right)^2+\cdots\right]
    \end{gather*}
    其中 $N=\frac{V}{3\pi^2}\left(\frac{2m}{\hbar}\varepsilon_F\right)^{\frac{2}{3}}$ 为电子总数,$\cdots$ 代表 $\frac{kT}{\varepsilon_F}$ 的高阶项.
\end{prob}
\begin{pf}
    
\end{pf}
\end{document}