\documentclass{assignment}
\ProjectInfos{高等热力学与统计物理}{PHYS2110}{2020-2021学年第二学期}{习题 IV}{截止时间:2021. 3. 30(周二)}{陈稼霖}{45875852}

\begin{document}
\begin{prob}
    计算下列系统内电子气体的 Fermi 能,Fermi 温度和 Fermi 速度.
    \begin{itemize}
        \item[1)] 室温下的金属钠:密度为 $0.97\mathrm{g}/\mathrm{cm}^3$,每个原子贡献一个传导电子,假设它们的能量和动量的关系为 $E=\frac{p^2}{2m}$,(即忽略晶格场对电子运动的影响).
        \item[2)] 天狼星的伴星(白矮星):其质量约为太阳质量的 $0.98$ 倍,半径约为太阳半径的 $0.0084$ 倍. 假设星体全部由氦组成.
    \end{itemize}
\end{prob}
\begin{sol}
    
\end{sol}

\begin{prob}
    证明非相对论简并电子气体的热力学函数是
    \begin{gather*}
        G=N\mu=N\varepsilon_F\left[1-\frac{1}{12}\pi^2\left(\frac{kT}{\varepsilon_F}\right)^2-\frac{1}{80}\pi^4\left(\frac{kT}{\varepsilon_F}\right)^4+\cdots\right]\\
        E=\frac{3}{5}N\varepsilon_F\left[1+\frac{5}{12}\pi^2\left(\frac{kT}{\varepsilon_F}\right)^2-\frac{1}{16}\pi^4\left(\frac{kT}{\varepsilon_F}\right)+\cdots\right]\\
        C_V=\frac{1}{2}N\pi^2\frac{k^2T}{\varepsilon_F}\left[1-\frac{3}{10}\pi^2\left(\frac{kT}{\varepsilon_F}\right)^2+\cdots\right]\\
        S=\frac{1}{2}N\pi^2\frac{k^2T}{\varepsilon_F}\left[1-\frac{1}{10}\left(\frac{kT}{\varepsilon_F}\right)^2+\cdots\right]
    \end{gather*}
    其中 $N=\frac{V}{3\pi^2}\left(\frac{2m}{\hbar}\varepsilon_F\right)^{\frac{2}{3}}$ 为电子总数,$\cdots$ 代表 $\frac{kT}{\varepsilon_F}$ 的高阶项.
\end{prob}
\begin{pf}
    
\end{pf}
\end{document}