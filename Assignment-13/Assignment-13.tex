\documentclass{assignment}
\ProjectInfos{高等热力学与统计物理}{PHYS2110}{2020-2021学年第二学期}{习题 XIII}{截止时间:2021. 6. 2(周二)}{陈稼霖}{45875852}

\begin{document}
\begin{prob}
    考虑硬球势
    \begin{itemize}
        \item[i)] 证明两体散射的 $s$ 波的相移为
        \[
            \delta_0(k)=-ka
        \]
        其中 $a$ 为硬球直径.
        \item[ii)] 证明对于自旋为零的 Bose 气体
        \[
            b_2=\frac{1}{\lambda^3}\left[2^{-5/2}-\frac{2a}{\lambda}+O\left(\frac{a^5}{\lambda^5}\right)\right]
        \]
        提示:对于硬球散射,轨道角动量为 $l$ 的相移的低能行为是
        \[
            \delta_l(k)\sim(ka)^{2l+1}.
        \]
    \end{itemize}
\end{prob}
\begin{sol}
    \begin{itemize}
        \item[i)] 
        两体散射体系的哈密顿量为
        \begin{align}
            H=-\frac{\hbar^2}{2m}\nabla^2+U(r),
        \end{align}
        其中直径为 $a$ 的硬球势为
        \begin{align}
            U(r)=\left\{\begin{array}{ll}
                +\infty,&r\leq a,\\
                0,&r>a.
            \end{array}\right.
        \end{align}
        设
        \begin{align}
            V(r)=\frac{2m}{\hbar^2}U(r),
        \end{align}
        则薛定谔方程
        \begin{align}
            -\frac{\hbar^2}{2m}\nabla^2\psi+U(r)\psi=E\psi,
        \end{align}
        化为
        \begin{align}
            \nabla^2\psi+[k^2-V(r)]\psi=0.
        \end{align}
        取沿粒子入射方向且通过散射中心的轴线为 $z$ 轴,对 $s$ 波,薛定谔方程的通解写为
        \begin{align}
            \psi(r,\theta)=\sum_lR_l(r)P_l(\cos\theta)
        \end{align}
        其径向波函数满足
        \begin{align}
            \frac{1}{r^2}\frac{\mathrm{d}}{\mathrm{d}r}\left(r^2\frac{\mathrm{d}R_l}{\mathrm{d}r}\right)+\left[k^2-V(r)-\frac{l(l+1)}{r}^2\right]R_l(r)=0
        \end{align}
        令 $R_l(r)=\frac{U_l(r)}{r}$,代入上式得
        \begin{align}
            \frac{\mathrm{d}^2U_l}{\mathrm{d}r^2}+\left[k^2-V(r)-\frac{l(l+1)}{r^2}\right]U_l(r)=0
        \end{align}
        对 $s$ 波,$l=0$,在 $r>a$ 处上述方程可化为
        \begin{align}
            \frac{\mathrm{d}^2U_l}{\mathrm{d}r^2}+k^2U_l(r)=0,\quad r>a
        \end{align}
        其解为
        \begin{align}
            U_l(r)=A_0\sin\left(kr+\delta_0\right),\quad r>a.
        \end{align}
        而在 $r\leq a$ 处,由于势能无穷大,故 $\psi=0$,即 $R_0(r)=0$,从而 $U_0(r)=0$.
        由于 $R_0(r)$ 在 $r=a$ 处连续,因此 $U_0(r)$ 在 $r=a$ 处连续,即
        \begin{gather}
            A\sin\left(ka+\delta_0\right)=0,\\
            \Longrightarrow\delta_0=-ka.
        \end{gather}
        \item[ii)] 对于自旋为零的 Bose 气体,根据 Beth-Uhlenbeck 公式
        \begin{align}
            b_2^S=b_2^{S(0)}+\frac{2^{3/2}}{\lambda^3}\left[\sum_{\text{even }l}e^{-\beta\varepsilon_B}+\frac{1}{\pi}\sum_{\text{even }l}(2l+1)\int_0^{\infty}\mathrm{d}k\,e^{-\frac{\beta}{m}\hbar^2k^2}\frac{\mathrm{d}\delta_l}{\mathrm{d}k}\right],
        \end{align}
        其中
        \begin{align}
            b_2^{S(0)}=\frac{1}{2^{5/2}\lambda^3},
        \end{align}
        由于气体原子的势能是硬球势,所以不存在束缚态,中括号中第一个求和式
        \begin{align}
            \sum_{\text{even }l}e^{-\beta\varepsilon_B}=0,
        \end{align}
        中括号中第二个求和式
        \begin{align}
            \notag\frac{1}{\pi}\sum_{\text{even }l}(2l+1)\int_0^{\infty}\mathrm{d}k\,e^{-\beta\hbar^2k^2}\frac{\mathrm{d}\delta_l}{\mathrm{d}k}=&\frac{1}{\pi}\int_0^{\infty}\mathrm{d}k\,e^{-\beta\hbar^2k^2}\frac{\mathrm{d}\delta_0}{\mathrm{d}k}+\frac{1}{\pi}\sum_{l=2,4,6,\cdots}(2l+1)\int_0^{\infty}\mathrm{d}k\,e^{-\beta\hbar^2k^2}\frac{\mathrm{d}\delta_l}{\mathrm{d}k}\\
            \notag\approx&\frac{1}{\pi}\int_0^{\infty}\mathrm{d}k\,e^{-\frac{\beta}{m}\hbar^2k^2}\frac{\mathrm{d}(-ka)}{\mathrm{d}k}+\frac{1}{\pi}\sum_{l=2,4,6,\cdots}(2l+1)\int_0^{\infty}\mathrm{d}k\,e^{-\frac{\beta}{m}\hbar^2k^2}\frac{\mathrm{d}(ka)^{2l+1}}{\mathrm{d}k}\\
            \notag=&-\frac{a}{\pi}\int_0^{\infty}\mathrm{d}k\,e^{-\frac{\beta}{m}\hbar^2k^2}+\frac{1}{\pi}\sum_{l=2,4,6,\cdots}a^{2l+1}(2l+1)^2\int_0^{\infty}\mathrm{d}k\,e^{-\frac{\beta}{m}\hbar^2k^2}k^{2l}\\
            \notag=&-\frac{a}{\pi}\frac{1}{2}\sqrt{\frac{m\pi}{\beta\hbar^2}}+\frac{1}{\pi}\sum_{l=2,4,6,\cdots}a^{2l+1}(2l+1)^2\int_0^{\infty}\mathrm{d}k\,e^{-\beta\hbar^2k^2}k^{2l}\\
            =&-\frac{a}{\sqrt{2}\lambda}+O\left(\frac{a^5}{\lambda^5}\right).
        \end{align}
        因此,
        \begin{align}
            b_2=\frac{1}{\lambda^3}\left[2^{-5/2}-\frac{2a}{\lambda}+O\left(\frac{a^5}{\lambda^5}\right)\right].
        \end{align}
    \end{itemize}
\end{sol}
\end{document}