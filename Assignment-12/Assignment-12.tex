\documentclass{assignment}
\ProjectInfos{高等热力学与统计物理}{PHYS2110}{2020-2021学年第二学期}{习题 XII}{截止时间:2021. 5. 25(周二)}{陈稼霖}{45875852}

\begin{document}
\begin{prob}
    求具有周期边界条件和仅有近邻相互作用 $u$ 的一维格气巨配分函数 $\mathcal{Q}(y)=0$ 的根 $y_1,y_2,\cdots,y_{\mathcal{N}}$,并讨论在 $u>0$ 和 $u<0$ 两种情况下根在复平面上的分布.
\end{prob}
\begin{sol}
    具有周期性边界条件且仅有近邻相互作用的一维 Ising 模型的配分函数为
    \begin{align}
        Q_I=\lambda_+^{\mathcal{N}}+\lambda_-^{\mathcal{N}},
    \end{align}
    其中
    \begin{align}
        \lambda_{\pm}=&\frac{1}{2}\left[x_I\left(y_I+\frac{1}{y_I}\right)\pm\sqrt{x_I^2\left(y_I-\frac{1}{y_I}\right)^2+\frac{4}{x_I^2}}\right],\\
        x_I=&e^{-\frac{1}{kT}\varepsilon},\\
        y_I=&e^{\frac{\mu H}{kT}}.
    \end{align}
    注意 $y_I$ 与格气的经典易逸度 $y$ 区分:
    \begin{align}
        y=e^{\frac{2}{kT}(\mu H+2\varepsilon)}.
    \end{align}
    对应的一维格气巨配分函数可表为
    \begin{align}
        \mathcal{Q}=Q_Ie^{\frac{\mathcal{N}(\mu H+\varepsilon)}{kT}}=(\lambda_+^{\mathcal{N}}+\lambda_-^{\mathcal{N}})e^{\frac{\mathcal{N}(\mu H+\varepsilon)}{kT}}=e^{\frac{\mathcal{N}\varepsilon}{kT}}y_I^{\mathcal{N}}(\lambda_+^{\mathcal{N}}+\lambda_-^{\mathcal{N}}),
    \end{align}
    其中有对应关系
    \begin{align}
        \varepsilon=\frac{u}{4}.
    \end{align}
    因为
    \begin{align}
        y_I\lambda_{\pm}=\frac{1}{2}\left[x_I\left(y_I^2+1\right)\pm\sqrt{x_I^2\left(y_I^2-1\right)^2+\frac{4y_I^2}{x_I^2}}\right],
    \end{align}
    所以
    \begin{align}
        \notag\mathcal{Q}=&e^{\frac{\mathcal{N}\varepsilon}{kT}}\sum_{n=0}^{\lfloor\mathcal{N}/2\rfloor}\left(\begin{matrix}
            \mathcal{N}\\
            2n
        \end{matrix}\right)\left[x_I(y_I^2+1)\right]^{\mathcal{N}-2n}\left[\sqrt{x_I^2(y_I^2-1)+\frac{4y_I^2}{x_I^2}}\right]^{2n}\\
        =&e^{\frac{\mathcal{N}\varepsilon}{kT}}\sum_{n=0}^{\lfloor\mathcal{N}/2\rfloor}\left(\begin{matrix}
            \mathcal{N}\\
            2n
        \end{matrix}\right)\left[x_I(y_I^2+1)\right]^{\mathcal{N}-2n}\left[x_I^2(y_I^2-1)+\frac{4y_I^2}{x_I^2}\right]^n.
    \end{align}
    注意到
    \begin{align}
        \label{1-YI-y}
        y_I=y^{1/2}e^{-\frac{2\varepsilon}{kT}},
    \end{align}
    故格气的巨配分函数 $\mathcal{Q}$ 是其经典易逸度 $y$ 的 $\mathcal{N}$ 次多项式,故 $\mathcal{Q}=0$ 应有 $\mathcal{N}$ 个根.
    \begin{gather}
        \mathcal{Q}=0,\\
        \Longrightarrow\lambda_+^{\mathcal{N}}=-\lambda_-^{\mathcal{N}},\\
        \Longrightarrow\lambda_+=\omega\lambda_-,
    \end{gather}
    其中
    \begin{align}
        \omega^{\mathcal{N}}=-1,
    \end{align}
    从而
    \begin{gather}
        x_I(y_I^2+1)+\sqrt{x_I^2(y_I^2-1)^2+\frac{4y_I^2}{x_I^2}}=\omega\left[x_I(y_I^2+1)-\sqrt{x_I^2(y_I^2-1)^2+\frac{4y_I^2}{x_I^2}}\right],\\
        \Longrightarrow y_I^4+2\left[1-\frac{(\omega+1)^2}{2\omega}\frac{x_I^4-1}{x_I^4}\right]y_I^2+1=0.
    \end{gather}
    由于 $\omega$ 的模为 $1$,不妨设 $\omega=e^{i\theta}$,则上面的方程可化为
    \begin{align}
        y_I^4+2\left[1-\left(1+\cos\theta\right)\frac{x^4-1}{x^4}\right]y_I^2+1=0.
    \end{align}

    当 $u>0$,$x_I<1$,$\left[1-\left(1+\cos\theta\right)\frac{x_I^4-1}{x_I^4}\right]<1$,方程有一对共轭负根 $y_{I,1},y_{I,2}$,由于$y_{I,1},y_{I,2}$乘积为 $1$,故这两根在单位圆上,而最终由式 \eqref{1-YI-y} 可以求得对应的 $y_1$ 和 $y_2$ 在以原点为圆心半径为 $e^{-\frac{4\varepsilon}{kT}}$ 的圆上,又由于 $\omega$ 可以有 $\mathcal{N}$ 个取值,但在 $\theta\rightarrow-\theta$ 对称下,方程实际上不变,故实际上有 $y_1,y_2,\cdots,y_{\mathcal{N}}$ 共计 $\mathcal{N}$ 个根在以原点为圆心,半径为 $e^{-\frac{4\varepsilon}{kT}}$ 的圆上.

    当 $u<0$,$x_I>1$,$\left[1-\left(1+\cos\theta\right)\frac{x_I^4-1}{x_I^4}\right]>1$,故方程有两个负实根 $y_{I,1},y_{I,2}$,同理可以由式 \eqref{1-YI-y} 和 $\omega$ 的不同取值得到 $y_1,y_2,\cdots,y_{\mathcal{N}}$ 共计 $\mathcal{N}$ 个实根.
\end{sol}

\begin{prob}
    当 $\mathcal{N}\rightarrow\infty$ 时,证明上一题讨论的一维格气的压强为
    \[
        \frac{P}{kT}=\ln\left[1+\frac{y}{x}+\sqrt{\left(1-\frac{y}{x}\right)^2+4y}\right]-\ln 2
    \]
    并求出粒子的数密度. 其中
    \[
        x=e^{\frac{u}{kT}}.
    \]
\end{prob}
\begin{pf}
    一维格气的巨配分函数为
    \begin{align}
        \mathcal{Q}=e^{\frac{\mathcal{N}\varepsilon}{kT}}y_I^{\mathcal{N}}(\lambda_+^{\mathcal{N}}+\lambda_-^{\mathcal{N}})
    \end{align}
    其中
    \begin{align}
        y_I\lambda_{\pm}=\frac{1}{2}\left[x_I(y_I^2+1)\pm\sqrt{x_I^2(y_I^2-1)^2+\frac{4y_I^2}{x_I^2}}\right].
    \end{align}
    由于 $\lambda_+>\lambda_-$,当 $\mathcal{N}\rightarrow\infty$,巨配分函数可近似为
    \begin{align}
        \mathcal{Q}\approx e^{\frac{\mathcal{N}\varepsilon}{kT}}y_I^{\mathcal{N}}\lambda_+^{\mathcal{N}}=e^{\frac{\mathcal{N}u}{kT}}y_I^{\mathcal{N}}\lambda_+^{\mathcal{N}}
    \end{align}
    由于
    \begin{align}
        x_I=&e^{-\frac{\varepsilon}{kT}}=x^{-1/4},\\
        y_I=&y^{1/2}e^{-\frac{2\varepsilon}{kT}}=y^{1/2}e^{-\frac{u}{2kT}}=y^{1/2}x^{-1/2},
    \end{align}
    故
    \begin{align}
        y_I\lambda_{\pm}=\frac{1}{2}\left[x^{-1/4}(yx^{-1}+1)\pm\sqrt{x^{-1/2}(yx^{-1}-1)^2+\frac{4yx^{-1}}{x^{-1/2}}}\right],
    \end{align}
    从而巨配分函数可表为
    \begin{align}
        \mathcal{Q}=x^{\mathcal{N}/4}y_I^{\mathcal{N}}\lambda_+^{\mathcal{N}}=\exp\left\{\frac{\mathcal{N}}{2}\left[\left(\frac{y}{x}+1\right)\right]+\sqrt{\left(1-\frac{y}{x}\right)^2+4y}\right\}.
    \end{align}
    一维格气的压强为
    \begin{align}
        \frac{P}{kT}=\frac{1}{\mathcal{N}}\ln\mathcal{Q}=\ln\left[1+\frac{y}{x}+\sqrt{\left(1-\frac{y}{x}\right)^2+4y}\right]-\ln 2.
    \end{align}
    % 对应 Ising 模型的磁化强度为
    % \begin{align}
    %     \notag M=&\frac{\langle N_{\uparrow}N_{\downarrow}\rangle}{\mathcal{N}}=\frac{1}{\mathcal{N}}\frac{kT}{\mu}\left(\frac{\partial}{\partial H}\ln Q_I\right)_T\approx\frac{kT}{\mu}\left(\frac{\partial}{\partial H}\ln\lambda_+\right)_T=\frac{kT}{\mu\lambda_+}\frac{\partial\lambda_+}{\partial H}\\
    %     \notag=&\frac{kT}{2\mu\lambda_+}\left[x_I\left(1-\frac{1}{y_I^2}\right)+\frac{x_I^2\left(1+\frac{1}{y_I^2}\right)\left(y_I-\frac{1}{y_I}\right)}{\sqrt{x_I^2\left(y_I-\frac{1}{y_I}\right)^2+\frac{4}{x_I^2}}}\right]\frac{\partial y_I}{\partial H}\\
    %     \notag=&\frac{kT}{2\mu\lambda_+}\left[x_I\left(1-\frac{1}{y_I^2}\right)+\frac{x_I^2\left(1+\frac{1}{y_I^2}\right)\left(1-\frac{1}{y_I^2}\right)}{\sqrt{x_I^2\left(1-\frac{1}{y_I^2}\right)^2+\frac{4}{x_I^2}}}\right]\frac{\mu}{kT}y_I\\
    %     =&\frac{y_I}{2\lambda_+}\left[x_I\left(1-\frac{1}{y_I^2}\right)+\frac{x_I^2\left(1+\frac{1}{y_I^2}\right)\left(1-\frac{1}{y_I^2}\right)}{\sqrt{x_I^2\left(1-\frac{1}{y_I^2}\right)^2+\frac{4}{x_I^2}}}\right].
    % \end{align}
    格气的粒子数密度为
    \begin{align}
        \rho=y\frac{\partial}{\partial y}\frac{P}{kT}=y\frac{\frac{1}{x}+\frac{-\frac{1}{x}\left(1-\frac{y}{x}\right)+4}{\sqrt{(1-\frac{y}{x})^2+4y}}}{1+\frac{y}{x}+\sqrt{\left(1-\frac{y}{x}\right)^2+4y}}.
    \end{align}
\end{pf}
\end{document}